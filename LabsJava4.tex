\documentclass[]{article}
\usepackage{lmodern}
\usepackage{amssymb,amsmath}
\usepackage{ifxetex,ifluatex}


\usepackage[utf8]{inputenc}
\usepackage[english,russian,ukrainian]{babel}

\usepackage{fixltx2e} % provides \textsubscript
\ifnum 0\ifxetex 1\fi\ifluatex 1\fi=0 % if pdftex
  \usepackage[T1]{fontenc}
  \usepackage[utf8]{inputenc}
\else % if luatex or xelatex
  \ifxetex
    \usepackage{mathspec}
  \else
    \usepackage{fontspec}
  \fi
  \defaultfontfeatures{Ligatures=TeX,Scale=MatchLowercase}
\fi
% use upquote if available, for straight quotes in verbatim environments
\IfFileExists{upquote.sty}{\usepackage{upquote}}{}
% use microtype if available
\IfFileExists{microtype.sty}{%
\usepackage{microtype}
\UseMicrotypeSet[protrusion]{basicmath} % disable protrusion for tt fonts
}{}
\usepackage[unicode=true]{hyperref}
\hypersetup{
            pdfborder={0 0 0},
            breaklinks=true}
\urlstyle{same}  % don't use monospace font for urls
\usepackage{graphicx,grffile}
\makeatletter
\def\maxwidth{\ifdim\Gin@nat@width>\linewidth\linewidth\else\Gin@nat@width\fi}
\def\maxheight{\ifdim\Gin@nat@height>\textheight\textheight\else\Gin@nat@height\fi}
\makeatother
% Scale images if necessary, so that they will not overflow the page
% margins by default, and it is still possible to overwrite the defaults
% using explicit options in \includegraphics[width, height, ...]{}
\setkeys{Gin}{width=\maxwidth,height=\maxheight,keepaspectratio}
\IfFileExists{parskip.sty}{%
\usepackage{parskip}
}{% else
\setlength{\parindent}{0pt}
\setlength{\parskip}{6pt plus 2pt minus 1pt}
}
\setlength{\emergencystretch}{3em}  % prevent overfull lines
\providecommand{\tightlist}{%
  \setlength{\itemsep}{0pt}\setlength{\parskip}{0pt}}
\setcounter{secnumdepth}{0}
% Redefines (sub)paragraphs to behave more like sections
\ifx\paragraph\undefined\else
\let\oldparagraph\paragraph
\renewcommand{\paragraph}[1]{\oldparagraph{#1}\mbox{}}
\fi
\ifx\subparagraph\undefined\else
\let\oldsubparagraph\subparagraph
\renewcommand{\subparagraph}[1]{\oldsubparagraph{#1}\mbox{}}
\fi

\date{}

\usepackage{multicol}

\usepackage{enumitem}
\makeatletter
\newcommand{\xslalph}[1]{\expandafter\@xslalph\csname c@#1\endcsname}
\newcommand{\@xslalph}[1]{%
    \ifcase#1\or а\or б\or в\or г\or д\or e\or є\or ж\or з\or i%
    \or й\or к\or л\or м\or н\or о\or п\or р\or с\or т%
    \or у\or ф\or х\or ц\or ч\or ш\or ю\or я\or аа\or бб\or вв %
    \else\@ctrerr\fi%
}
\AddEnumerateCounter{\xslalph}{\@xslalph}{m}
\makeatother


\begin{document}


Збірник задач для вивчення мови Java


\section{4. Створення власних класів}

\subsection{Аудиторні задачі}
\begin{enumerate}
\def\labelenumi{1.\arabic{enumi}.}

\item
Створіть клас у пакеті. Створіть екземпляр свого класу поза цим пакетом.
\item
Покажіть, що захищені методи мають доступ до пакетів, але не є загальнодоступними.
\item
 Створіть клас із загальнодоступними, приватними, захищеними та доступом за замовченням члени та функціями-членами. Створіть об’єкт цього класу і подивіться, які повідомлення компілятора ви отримуєте, коли намагаєтесь отримати доступ до всіх членів класу. Майте на увазі, що класи в одному каталозі є частиною пакета "за замовчуванням".
\item
 Створіть клас із захищеними даними. Створіть другий клас у тому ж файлі методом, який маніпулює захищеними даними першого класу.
\item
 Створіть два пакети: debug та debugoff, що містять ідентичний клас із методом debug(). Перша версія відображає аргументи типу String
(довільну кількість) на консолі, друга не відображає. Використовуйте static import, щоб імпортувати клас у тестову програму, і продемонструйте ефект заміни класу щоб відображати та не відображати аргументи.

\end{enumerate}

\subsection{Власні класи}
Створити класи, специфікації яких наведені нижче.
Визначити для кожного класу конструктори та метод toString(). Для кожного члену класу Type - публічні методи setType(), getТype(). 
Визначити додатково методи в класі, що створює масив об'єктів. Задати критерій вибору даних і вивесті ці дані на консоль. 
У кожному класі, що містить дані, має бути оголошено декілька конструкторів.
\begin{enumerate}
\def\labelenumi{2.\arabic{enumi}.}
\item
Student: id, Прізвище, Ім'я, По батькові, Дата народження, адресу, телефон, Факультет, Курс, Группа.
Створити масив об'єктів. Вивести: a) список студентів заданого факультету; b) списки студентів для кожного факультету ікурса; c) список студентів, які народилися після заданого року; d) список навчальної групи.
\item Customer: id, Прізвище, Ім'я, По батькові, Адреса, Номер кредитної картки, Номер банківського рухунку.
Створити масив об'єктів. Вивести: a) список покупців в алфавітном порядку; b) список покупців, у яких номер кредитної картки знаходиться в заданому інтервалі.
\item Patient: id, Прізвище, Ім'я, По батькові, адресу, телефон, Номер медичної карти, Діагноз.
Створити масив об'єктів.. Вивести: a) список пацієнтів, що мають цей діагноз; b) список пацієнтів, номер медичної карти яких знаходиться в 
заданому інтервалі.
\item Abiturient: id, Прізвище, Ім'я, По батькові, адресу, телефон, Оцінки. 
Створити масив об'єктів. Вивести: a) список абітурієнтів, що мають незадовільні оцінки; 
b) список абітурієнтів, у яких сума балів вище заданої; c) вибрати заданий число n абітурієнтів, що мають найвищу суму балів (вивести також повний список абітурієнтів, що мають напівпрохідну суму).
\item Book: id, Назва, Автор (и), Видавництво, Рік видання, Кількість страниць, Ціна, Тип переплета.
Створити масив об'єктів. Вивести: a) список книг заданого автора; b) список книг заданого видавництва; 
c) список книг, надрукованих після заданого року.
\item House: id, Номер квартири, Площа, Поверх, Кількість кімнат, Вулиця, Тип будівлі, Термін експлуатаціі.
Створити масив об'єктів. 
Вивести: a) список квартир, що мають задане число кімнат; b) список квартир, що мають задане число кімнат в даному проміжку поверхів; c) список квартир, які мають площу, яка перевищує задану.
\item
Phone: id, Прізвище, Ім'я, По батькові, Адреса, Номер кредитної картки, Дебет, Кредит, Час міських імеждугородних разговоров.Создать масив об'єктів. Вивести: a) відомості обабонентах, укоторих час внутрішньоміських розмов перевищує ліміт; b) відомості обабонентах, які користувалися міжміським зв'язком; c) відомості обабонентах валфавітном порядку.
\item
Car: id, Марка, Модель, Рік випуску, Колір, Ціна, Реєстраційний номер. 
Створити масив об'єктів. Вивести: a) список автомобілів заданої марки; b) список автомобілів заданої моделі, які експлуатуються більше n років; c) список автомобілів заданого року випуску, ціна яких більше вка-занной.

\item
Product: id, Найменування, UPC, Виробник, Ціна, Термін зберігання, Колічество. 
Створити масив об'єктів. Вивести: a) список товарів для заданого найменування; b) список товарів для заданого найменування, ціна яких непревос-ходить задану; c) список товарів, термін зберігання яких більше заданого.

\item Train: Пункт призначення, Номер поїзда, Час відправлення, Число місць (загальних, купе, плацкарт, люкс).
Створити масив об'єктів. 
Вивести: a) список поїздів, які прямують дозаданного пункту призначення; b) список поїздів, які прямують дозаданного пункту призначення іотправ-рами після заданого години; c) список поїздів, які відправляються дозаданного пункту призначення і мають спільні місця.

\item
Bus: Прізвище та ініціали водія, Номер автобуса, Номер маршруту, Марка, Рік початку експлуатації, Пробег.
Створити масив об'єктів. Вивести: a) список автобусів для заданого номера маршруту; b) список автобусів, які експлуатуються більше заданого терміну; c) список автобусів, пробіг у яких більше заданої відстані.

\item
Airline: Пункт призначення, Номер рейсу, Тип літака, Час вильоту, Дні неделі. Створити масив об'єктів. Вивести: a) список рейсів для заданого пункту призначення; b) список рейсів для заданого дня тижня; c) список рейсів для заданого дня тижня, час вильоту для яких більше заданого.

\end{enumerate}

\subsection{Математичні класи}

Реалізувати методи додавання, віднімання, множення і ділення об'єктів (для тих класів, об'єкти яких можуть підтримувати арифметичні дії).
\begin{enumerate}
\def\labelenumi{3.\arabic{enumi}.}
\item
Визначити клас РаціональнийДріб у вигляді пари чисел $m$ і $n$. Оголосити та ініціалізувати масив із $k$ дробів, ввести / вивести значення для масиву дробів. Створити масив таких об'єктів і та обчисліть їх суму.
\item
Визначити клас Комплекс. Створити масив / список / множину розмірності $n$ із комплексних координат. Передати його в метод, який виконає додавання / множення його елементів.
\item
 Визначити клас Квадратне рівняння. Реалізувати методи для пошуку коренів, екстремумів, а також інтервалів убування / зростання. Створити масив / список / множину об'єктів і визначити найбільші і найменші значення коріння.
\item
Визначите клас Поліном ступеня $n$. Оголосити масив / список / множину із $m$ полиномов і визначить суму поліномів масиву.
\item
Визначить клас Інтервал с урахуванням включення / невключення. Створити методи по знаходженню перетину і об'едінанню інтервалів, причому інтервали, що немають спільних точок, перетинатися /об'єднуватися неможуть. Оголосити масив / список / множину з n інтервалів і визначить відстань між найбільш віддаленими кінцями.
\item
Визначить клас Точка на площині (в просторі) та в часі. Задати рух точки у певному напрямку. Створити методи по знаходженню швидкості та прискорення точки. Перевірити для двох точок можливість перетину траєкторій. Визначити відстань між двома точками в заданний момент часу.
\item
Визначить клас Трикутник на площині. Визначити площу і периметр трикутника. Створити масив / список / множину об'єктів і підрахувати кількість трикутників різного типу (рівносторонній, равнобедрений, прямокутний, довільний). Визначити для кожної групи найбільший і найменшій по площаді (периметру) об'єкт.
\item

Визначить клас Чотирикутник на площині. Визначити площу і періметр чотирикутника. Створити масив / список / множину об'єктів і підрахуйте кількість чотирикутників різного типу (квадрат, прямокутник, ромб, довільний). Визначити для кожної групи найбільший і найменший за площею (периметром) об'єкт.
\item
Визначить клас Коло на площині. Визначити площу і периметр. Створити масив / список / множину об'єктів і знайдіть групи кіл, центри яких лежать на одной прямий. Визначити найбільший і найменшій по площині (периметру) об'єкт
\item  Визначить клас Пряма на площині та просторі. Визначити точки перетину прямої з вісями координат. Визначити координати перетину двох прямих. Створити масив / список / множину об'єктів і определіть групи паралельних прямих
\end{enumerate}

\subsection{Масиви класів}  

\begin{enumerate}
\def\labelenumi{4.\arabic{enumi}.}
\item
 Створіть клас під назвою ConnectionManager, який керує фіксованим масивом об’єктів Connection. Клієнт -програміст не повинен мати можливість явно створювати об'єкти Connection, але може отримати їх лише статичним методом у ConnectionManager. Коли у ConnectionManager закінчуються об'єкти, він завершує роботу. 
\end{enumerate}

\begin{enumerate}
\def\labelenumi{4.\arabic{enumi}.}
\item
Визначити клас Поліном c коефіцієнтами типу РаціональнийДріб. Оголосити масив / список / множину із n полиномов і определіть суму поліномів масиву.
\item
Визначити клас Пряма на площині (в просторі), параметри якої задаються з допомогою РаціональногоДробу. Визначити точки перетину прямою з вісямі координат. Визначити координати перетину двох прямих. Створити масив / список / множину об'єктів і визначить групи паралельних прямих.
\item
Визначити клас Поліном скоеффіціентамі типу КомплекснеЧисло. Оголосити масив / список  із m поліномів і визначить суму поліномів массіва.
\item
Визначить клас Дріб у вигляді пари (m, n) з коеффіціентамі типу КомплекснеЧисло. Оголосити і форматувати масив із kдробей, ввести / вивести значення для масиву дробів. Створити масив / список /  об'єктів і передать його в метод, який змінює кожен елемент мас-иву за індексом шляхом додавання наступного за ним елементу.
\item
Визначить клас Комплекс, дійсна і уявна частина якої представлені у вигляді РаціональногоДробу. Створити масив / список  розмірності n із комплексних координат. Передати його вметод, котрий виконує додавання / множення його елементов.
\item
Визначить клас Окружність на площин, координати центру якої задаються з допомогою РаціональногоДробу. Визначити площу і периметр. Створити масив / список / множину об'єктів і визначити групи кіл, центри яких лежать на одной прямий. Визначити найбільший і найменьшій по площині (периметру) об'єкт.
\item
Визначить клас Точка в просторі, координати якої задаються з допомогою РаціональногоДробу. Створити методи по визначенню відстані між точками і відстані до початку координат. Перевірити для трьох точок можливість знаходження на одной прямий.
\item
Визначить клас Точка в просторі, координати якої задаються з допомогою КомплексногоЧисла. Створити методи по визначенню відстані між точками і до початку координат.
\item
Визначить клас Трикутник на площині, вершини якого мають тип Точка. Визначити площу і периметр трикутника. Створити масив / список / множину об'єктів і знайдіть кількість трикутників різного типу (рівносторонній, рівнобедрений, прямокутний, довільний). Визначити для кожної групи найбільший та найменьшій по площаді.
\item
Визначити клас Квадратне рівняння для дійсних та комплексних коренів. Реалізувати методи для пошуку коренів, екстремумів, а також інтервалів убування / зростання. Створити масив / список / множину об'єктів і визначити найбільші і найменші значення коріння. Створити список Рівнянь та розвяжить систему квадратних нервностей
\item
Визначить клас Інтервал с урахуванням включення / невключення. Створити методи по знаходженню перетину і об'єднанню інтервалів, причому інтервали, що немають спільних точок, перетинатися /обєднуватися неможуть. Оголосити масив / список / множину з n інтервалів і визначить відстань між найбільш віддаленими кінцями. Методи роботи зі списком інтервалів: додавання інтервалу, групування і т.ін. 
\item
Визначить клас Точка на площині (в просторі) та в часі Задати рух точки у певному напрямку. Створити методи по знаходженню швидкості та прискорення точки. Перевірити для двох точок можливість перетину траєкторій. Визначити відстань між двома точками в заданний момент часу.
\item
Визначить клас Вектор. Реалізувати методи інкремента, декременту, індексування. Визначити масив з m об'єктів. Кожну з пар векторів передати в методи, які повертають їх скалярний добуток і довжини. Обчислити і вивести кути між векторами.
\item
Визначить клас Вектор. Реалізувати методи для обчислення модуля вектора, скалярного твори, додавання, віднімання, множення наконстанту. Оголосити масив об'єктів. Написати метод, який для заданої пари векторів буде визначати, чи є вони колінеарними або ортогональними.
\item
Визначить клас Вектор в R3. Реалізувати методи для перевірки векторів на ортогональность, перевірки перетину неортогональних векторів, порівняння векторів. Створити масив із m об'єктів. Визначити компланарні вектори.
\item
Визначить клас БулеваМатріця (BoolMatrix). Реалізувати методи для логічного додавання (диз'юнкції), множення і інверсії матриць. Реалізувати методи для підрахунку числа одиниць в матриц і впорядкування рядків в лексикографічному порядку.
\item
Побудуйте клас БулевВектор (BoolVector). Реалізувати методи для виконання порозрядних кон'юнкції, диз'юнкції і заперечення векторів, а також підрахунку числа одиниць і нулів у векторі.
\item
Визначить клас МножинаСимволів. Реалізувати методи для визначення приналежності заданого елемента множині; перетину, об'єднання, різниці двох множин. Створити методи додавання, віднімання, множення (перетину), індексування, присвоєння. Створити масив об'єктів і передавати пари об'єктів в метод іншого класу, який будує множину, що складається із елементов, що входять тільки в одну із заданих множин.
\item
Визначить клас НелінійнеРівняння для однієї змінної. Клас дозволяє задавати інтервал де шукається корінь та створювати рівняння як поліном 5-го ступеню та від функцій сінус та експонента. Реалізувати метод визначення коренів методом бієкції.
\item
Визначить клас ВизначенийІнтеграл з аналітичної підінтегральної функції. Клас дозволяє задавати інтервал інтегрування та створювати рівняння як поліном 5-го ступеню та від функцій косинус, корінь та логарифм. Створити методи для обчислення значення за формулою лівих прямокутників, за формулою правих прямокутників, формулою середніх прямокутників, по формулі трапецій.
\item
Визначити клас Масив. Створити методи сортування: обмінне сортування (метод бульбашки); обмінне сортування «Шейкер-сортування», сортування за допомогою вибору (метод простого вибору), сортування вставками: метод хешування (сортування з обчисленням адреси), сортування вставками (метод простих вставок), сортування бінарним злиттям, сортування Шелла (сортування з спадаучим кроком)
\end{enumerate}

\section{5. Ієрархії класів. Інтерфейси. Внутрішні класи}

Створити додаток, яке задовольняє вимогам, наведеним в завданні. Спадкування застосовувати тільки втих завданнях, в яких це логічно обґрунтоване. Аргументувати приналежність класу кожного створюваного методу ікорректно перевизначити для кожного класу методи equals (), hashCode (), toString ().
\begin{enumerate}
\def\labelenumi{5.\arabic{enumi}.}
\item
Создать об'єкт класу Текст, використовуючи класи Речення, Слово. Методи: доповнити текст, вивести на консоль текст, заголовок тексту.
\item
Создать об'єкт класу Автомобіль, використовуючи класи Колесо, Двигун. Методи: їхати, заправлятися, міняти колесо, вивести наконсоль марку автомобіля.
\item
Создать об'єкт класу Літак, використовуючи класи Крило, Шасі, Двигун. Методи: літати, задавати маршрут, вивести наконсоль маршрут.
\item
Создать об'єкт класу Держава, використовуючи класи Область, Район, Місто. Методи: вивести на консоль столицю, кількість областей, площа, обласні центри.
\item
Создать об'єкт класу Планета, використовуючи класи Материк, Океан, Острів. Методи: вивести на консоль назву материка, планети, кількість материків.
\item
Создать об'єкт класу ЗорянаСистема, використовуючи класи Планета, Зірка, Місяць. Методи: вивести на консоль кількість планет системи, назва зірки, додавання планети в систему.
\item
Создать об'єкт класу Комп'ютер, використовуючи класи Вінчестер, Дисковод, Оперативна пам'ять, Процесор. Методи: включити, вимкнути, перевірити на віруси, вивести наконсоль розмір вінчестера.
\item
Создать об'єкт класу Квадрат, використовуючи класи Точка, Відрізок.Методи: завдання розмірів, розтягнення, стиснення, поворот, зміна кольору.
\item
Создать об'єкт класу Коло, використовуючи класи Точка, Окружність. Методи: завдання розмірів, зміна радіуса, визначення приналежності точки даного кола.
\item
Создать об'єкт класу Щеня, використовуючи класи Тварина, Собака. Методи: вивести наконсоль ім'я, подати голос, стрибати, бігати, кусати.
\item
Создать об'єкт класу Квочка, використовуючи класи Птах, Зозуля. Методи: літати, співати, нести яйця, висиджувати пташенят.
\item
Создать об'єкт класу ТекстовийФайл, використовуючи класи Файл, Директорія. Методи: створити, перейменувати, вивести наконсоль вміст, доповнити, видалити.
\item
Создать об'єкт класу ОдномірнийМасив, використовуючи класи Масив, Елемент. Методи: створити, вивести наконсоль, виконати операції (скласти, відняти, помножити).
\item
Создать об'єкт класу ПростийДріб, використовуючи клас Чісло. Методи: виведення на екран, додавання, віднімання, множення, ділення.
\item
Создать об'єкт класу Будинок, використовуючи класи Вікно, Двері. Методи: закрити наключ, вивести наконсоль кількість вікон, дверей.
\item
Создать об'єкт класу Квітка, використовуючи класи Пелюсток, Бутон. Методи: розквітнути, зів'яли, вивести наконсоль колір бутона.
\item
Создать об'єкт класу Дерево, використовуючи класи Лист, Гілка. Методи: зацвісти, листопад, покритися інеєм, пожовтіти лист.
\item
Создать об'єкт класу Піаніно, використовуючи класи Кнопка, Педаль. Методи: налаштувати, грати напіаніно, натискати клавішу.
\item
Создать об'єкт класу Фотоальбом, використовуючи класи Фотографія, Сторінка. Методи: задати назву фотографії, доповнити фотоальбом фотографією, вивести на консоль кількість фотографій.
\item
Создать об'єкт класу Рік, використовуючи класи Місяць, День. Методи: задати дату, вивести на консоль день тижня по заданній даті, розрахувати кількість днів, місяців взаданном часовому проміжку.
\item
Создать об'єкт класу Доба, використовуючи класи Час, Хвилина. Методи: вивести наконсоль поточний час, розрахувати час доби (ранок, день, ве-чер, ніч).
\item
Создать об'єкт класу Птах, використовуючи класи Крила, Клюв.Методи: літати, сідати, харчуватися, атакувати.
\item
Створити об'єкт класу Хижак, використовуючи класи Кігті, Зуби. Методи: гарчати, бігти, спати, добувати їжу.
\end{enumerate}

\subsection{Поліморфізм}
Створити консольний додаток, яке задовольняє наступним вимогам:
• Використовувати можливості ООП: класи, спадкування, поліморфізм, інкапсуляція.
• Кожен клас повинен мати відображає сенс назву іінформатівний склад.
• Спадкування має застосовуватися тільки тоді, коли це має сенс.
• При кодуванні повинні бути використані угоди обоформленіі коду java code convention.
• Класи повинні бути грамотно розкладені по пакетах.
• Консольне меню повинно бути мінімальним.
\begin{enumerate}
\def\labelenumi{6.\arabic{enumi}.}
\item
Квіткарня. Визначити ієрархію квітів. Створити кілька об'єктів-квіток. Зібрати букет (використовуючи аксесуари) з визначенням його вартості. Провести сортування квітів в букеті на  основі рівня свіжості. Знайти квітку в  букеті, що відповідає заданому діапазону довжини стебля. 
\item
Новогодній подарунок. Визначити ієрархію цукерок та інших солодощів. Створити кілька об'єктів-цукерок. Зібрати дитячий подарунок з урахуванням його ваги. Провести сортування цукерок в подарунок на основі одного з параметрів. Знайти цукерку в подарунок, відповідну заданому діапазону вмісту цукру.
\item
Домашні електроприлади. Визначити ієрархію електроприладів. Ввімкнути деякі в розетку. Підрахувати споживану потужність. Провести сортування приладів в квартирі на базі потужності. Знайти прилад в квартирі, що відповідає заданому діапазону параметрів.
\item
Шеф-кухар. Визначити ієрархію овочів. Зробити салат. Підрахувати калорійність. Провести сортування овочів для салату на основе одного із параметров. Знайти овочі в салаті, відповідні заданому діапазону калорійності.
\item
Звукозапіс. Визначити ієрархію музичних композицій. Записати на діск збірку. Підрахувати тривалість. Провести перестановку композицій  на диску на базі належності стилю. Знайти композицію, відповідну заданому діапазону довжини треків.
\item
Камені. Визначити ієрархію дорогоцінних і полудрагоцінних каменів. Відібрати камені для намиста. Підрахувати загальну вагу (в каратах) і вартість. Провесті сортування каменів намиста на базі цінності. Знайти камені в намисті, відповідні заданому діапазону параметрів прозорості.
\item
Мотоцікліст. Визначити ієрархію амуніції. Екіпірувати мотоциклиста. Підрахувати вартість. Провести сортування амуніції по весу. Знайти елементи амуніції, відповідні заданому діапазону параметрів ціни.
\item
Транспорт. Визначити ієрархію рухомого складу залізничного транспорту. Створити пасажирський поїзд. Підрахувати загальну чисельність пасажирів і багажу. Провести сортування вагонів поїзда на базі рівню комфортності. Знайти в потягу вагони, відповідні заданому діапазону параметрів кількості пасажирів.
\item  Авіакомпанія. Визначити ієрархію літаків. Створити авіакомпанію. Порахувати загальну місткість і вантажопідйомність. Провести сортування літаків компанії за дальністю польоту. Знайти літак в компаніі, що відповідає заданому діапазону параметрів споживання пального.
\item  Таксопарк. Визначити ієрархію легкових автомобілів. Створити таксо-парк. Підрахувати вартість автопарку. Провести сортування автомобілів парку по  розходу палива. Знайти автомобіль в компаніі, що відповідає заданому діапазону параметрів швидкості.
\item  Страхування. Визначити ієрархію страхових зобов'язань. Зібрати із зобвязань дериватив. Підрахувати вартість. Провести сортування зобовязань у в деріватіві на базі зменшення ступеня ризику. Знайти зобов'язання в деривативах, що  відповідне заданому діапазону параметрів.
\item  Мобільная зв'язок. Визначити ієрархію тарифів мобільної компанії. Створити список тарифів компанії. Підрахувати загальну кількість клієнтів. Провести сортування тарифів на базі розміру абонентської плати. Знайти тариф в компаніі, що відповідає заданому діапазону параметрів.
\item  Фургон кави. Завантажити фургон певного обєму вантажем на певну суму із різних сортів кави, що знаходяться, до того-ж, в різних фізичних станах (зерно, мелену, розчинну в банках і пакетиках). Враховувати обсяг кави разом з упаковкою. Провести сортування товарів на основі співвідношення ціни і ваги. Знайти в фургоні товар, відповідний заданому діапазону параметрів якості.
\item  Ігрова кімната. Підготувати ігрову кімнату для дітей різних вікових груп. Іграшок повинно бути фіксована кількість в межах виділеної суми грошей. Повинні зустрічатися іграшки родинних груп: маленькі, середні і великі машини, ляльки, м'ячі, кубики. Провести сортування іграшок в кімнаті по одному з параметрів. Знайти іграшки в кімнаті, відповідні заданому діапазону параметрів.
\item  Податки. Визначити кількість і суму податкових виплат фізичної особи заздалегідь з урахуванням доходів сосновной і додаткового місць роботи, авторських винагород, продажу майна, отримання вподарок грошових сум і майна, переказів з-за кордону, пільг на дітей і матеріальної допомоги. Провести сортування податків за сумою.
\item  Рахунки. Клієнт може мати кілька рахунків в банках. Враховувати можливість блокування / розблокування рахунку. реалізувати пошук

\end{enumerate}

\subsection{Інтерфейси}
 Реалізувати інтерфейси, також використовувати успадкування та поліморфізм для наступних предметних областей:
\begin{enumerate}
\def\labelenumi{7.\arabic{enumi}.}

\item  interface Видання - abstract class - Книга - class - Довідник і Енціклопедія.
\item  interface Абітурієнт  - abstract class - Студент  - class - Студент-Заочник.
\item  interface Співробітник  - class - Інженер  - class - Керівник.
\item  interface Будівля  - abstract class - Громадська Будівля  - class - Театр.
\item  interface Mobile  - abstract class - CoolFirm Mobile  - class - Model.
\item  interface Корабель  - abstract class - Військовий Корабель  - class - Авіаносець.
\item  interface Лікар  - class - Хірург  - class - Нейрохірург.
\item  interface Корабель  - class - Вантажний корабель  - class - Танкер.
\item  interface Меблі  - abstract class - Шафа  - class - Книжкова Шафа.
\item  interface Фільм  - class - Вітчизняний Фільм  - class - Комедія.
\item  interface Тканина  - abstract class - Одежа - class - Костюм.
\item  interface Техніка  - abstract class - Плеєр  - class - Відеоплеєр.
\item  interface Транспортне Засіб  - abstract class - Громадський Транспорт  - class - Трамвай.
\item  interface Пристрій Печатки  - class - Принтер  - class - Лазерний Принтер.
\item  interface Папір  - abstract class - Зошит  - class - Зошит Для Малювання.
\item  interface Джерело Світла  - class - Лампа  - class - Настільна Лампа.
\end{enumerate}

\subsection{Внутрішні класи}
\begin{enumerate}
\def\labelenumi{8.\arabic{enumi}.}

\item  Создать клас Payment з внутрішнім класом, за допомогою об'єктів якого можна сформувати покупку з декількох товарів.
\item  Создать клас Account свнутреннім класом, за допомогою об'єктів якого можна зберігати інформацію про всі операції сосчетом (зняття, платежі, надходження).
\item  Создать клас ЗачетнаяКніжка свнутреннім класом, за допомогою об'єктів якого можна зберігати інформацію про сесії, заліках, іспитах.
\item  Создать клас Department з внутрішнім класом, за допомогою об'єктів якого можна зберігати інформацію про всі посадах відділу і про всіх співробітників, коли-небудь займали конкретну посаду.
\item  Создать клас Catalog з внутрішнім класом, за допомогою об'єктів якого можна зберігати інформацію про історію видач книги читачам.
\item  Создать клас Європа з внутрішнім класом, за допомогою об'єктів якого можна зберігати інформацію про історію зміни територіального поділу на держави.
\item  Создать клас City з внутрішнім класом, за допомогою об'єктів якого можна зберігати інформацію про проспектах, вулицях, площах.
\item  Создать клас BlueRayDisc з внутрішнім класом, за допомогою об'єктів якого можна зберігати інформацію про каталогах, підкаталогах і записах.
\item  Создать клас Mobile з внутрішнім класом, за допомогою об'єктів якого можна зберігати інформацію про моделі телефонів і їх властивості.
\item  Создать клас Художня Виставка з внутрішнім класом, за допомогою об'єктів якого можна зберігати інформацію про картини, авторів і часу проведення виставок.
\item  Создать клас Календар з внутрішнім класом, за допомогою об'єктів якого можна зберігати інформацію про вихідні та святкові дні.
\item  Создать клас Shop з внутрішнім класом, за допомогою об'єктів якого можна зберігати інформацію про відділи, товари і послуги.
\item  Создать клас довідкового центру Cлужба Суспільні Tранспорт з внутрішнім класом, за допомогою об'єктів якого можна зберігати інформацію про час, лініях маршрутів і вартості проїзду.
\item  Создать клас Computer з внутрішнім класом, за допомогою об'єктів якого можна зберігати інформацію про операційну систему, процесорі і оперативної пам'яті.
\item  Создать клас Park з внутрішнім класом, за допомогою об'єктів якого можна зберігати інформацію про атракціони, часу їх роботи і вартості.
\item  Создать клас Cinema з внутрішнім класом, за допомогою об'єктів якого можна зберігати інформацію про адреси кінотеатрів, фільмах і часу початку сеансів
\end{enumerate}

\subsection{Виключення}

Для всіх класів, які робилися в домашнх  вправах:

а) зробити обробку стандартних виключень для всіх ситуацій отримання вводу
(унеможливьте будь-який некоректний ввод); 

б) зробити власні виключення для обробки неможливих ситуацій згідно програми та обробить їх в головному класі.


\end{document}


