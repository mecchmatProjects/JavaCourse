\documentclass[]{article}
\usepackage{lmodern}
\usepackage{amssymb,amsmath}
\usepackage{ifxetex,ifluatex}


\usepackage[utf8]{inputenc}
\usepackage[english,russian,ukrainian]{babel}

\usepackage{fixltx2e} % provides \textsubscript
\ifnum 0\ifxetex 1\fi\ifluatex 1\fi=0 % if pdftex
  \usepackage[T1]{fontenc}
  \usepackage[utf8]{inputenc}
\else % if luatex or xelatex
  \ifxetex
    \usepackage{mathspec}
  \else
    \usepackage{fontspec}
  \fi
  \defaultfontfeatures{Ligatures=TeX,Scale=MatchLowercase}
\fi
% use upquote if available, for straight quotes in verbatim environments
\IfFileExists{upquote.sty}{\usepackage{upquote}}{}
% use microtype if available
\IfFileExists{microtype.sty}{%
\usepackage{microtype}
\UseMicrotypeSet[protrusion]{basicmath} % disable protrusion for tt fonts
}{}
\usepackage[unicode=true]{hyperref}
\hypersetup{
            pdfborder={0 0 0},
            breaklinks=true}
\urlstyle{same}  % don't use monospace font for urls
\usepackage{graphicx,grffile}
\makeatletter
\def\maxwidth{\ifdim\Gin@nat@width>\linewidth\linewidth\else\Gin@nat@width\fi}
\def\maxheight{\ifdim\Gin@nat@height>\textheight\textheight\else\Gin@nat@height\fi}
\makeatother
% Scale images if necessary, so that they will not overflow the page
% margins by default, and it is still possible to overwrite the defaults
% using explicit options in \includegraphics[width, height, ...]{}
\setkeys{Gin}{width=\maxwidth,height=\maxheight,keepaspectratio}
\IfFileExists{parskip.sty}{%
\usepackage{parskip}
}{% else
\setlength{\parindent}{0pt}
\setlength{\parskip}{6pt plus 2pt minus 1pt}
}
\setlength{\emergencystretch}{3em}  % prevent overfull lines
\providecommand{\tightlist}{%
  \setlength{\itemsep}{0pt}\setlength{\parskip}{0pt}}
\setcounter{secnumdepth}{0}
% Redefines (sub)paragraphs to behave more like sections
\ifx\paragraph\undefined\else
\let\oldparagraph\paragraph
\renewcommand{\paragraph}[1]{\oldparagraph{#1}\mbox{}}
\fi
\ifx\subparagraph\undefined\else
\let\oldsubparagraph\subparagraph
\renewcommand{\subparagraph}[1]{\oldsubparagraph{#1}\mbox{}}
\fi

\date{}
\usepackage{multicol}
\usepackage{enumitem}
\makeatletter
\newcommand{\xslalph}[1]{\expandafter\@xslalph\csname c@#1\endcsname}
\newcommand{\@xslalph}[1]{%
    \ifcase#1\or а\or б\or в\or г\or д\or e\or є\or ж\or з\or i%
    \or й\or к\or л\or м\or н\or о\or п\or р\or с\or т%
    \or у\or ф\or х\or ц\or ч\or ш\or ю\or я\or аа\or бб\or вв %
    \else\@ctrerr\fi%
}
\AddEnumerateCounter{\xslalph}{\@xslalph}{m}
\makeatother
\begin{document}
Збірник задач для вивчення мови Java

  \section{6. Рядки. Регулярні вирази. Текстові файли}

Класи String, StringBuffer і StringTokenizer. 

\begin{enumerate}
\def\labelenumi{\arabic{enumi}.}
\item 
 Напишіть програму, яка підраховує число слів у тексті. Розгляньте випадки, коли текст визначається в самій програмі і вводиться з командного рядка при запуску програми на виконання.
\item  Напишіть програму, що записує текст у зворотному порядку. Розгляньте випадки, коли текст визначається в самій програмі і вводиться з командного рядка при запуску програми на виконання.
\item  Напишіть програму, яка в даному рядку замінить всі слова “small” на “very large”.
\item  Напишіть програму, яка в даному рядку видалить артиклі “a” та "an" .
\item  Напишіть програму, яка в перед кожним словами “small” та "large" додасть слово “very”.
\item Напишіть програму, яка в кінець рядка додасть рядок  новий рядок “, that we use to ilustrate the methods of class StringBuffer” не створюючи нового рядка. 
\item Напишіть програму, яка підраховує число символів, слів, знаків пунктуації та речень у тексті. Розгляньте випадки, коли текст визначається в самій програмі і вводиться з командного рядка при запуску програми на виконання.
\item Напишіть програму, яка перевіряє, чи певне слово міститься  в даному тексті. Розгляньте випадки, коли текст визначається в самій програмі і вводиться з командного рядка при запуску програми на виконання. Програма повинна розпізнавати слово незалежно від регістру, в якому воно записане. Розгляньте випадки залежно/незалежно від регістру.
\item Напишіть програму, яка обчислює відносну частоту появи кожного символу в даному тексті, включаючи знаки пунктуації і пробіли.
\item Напишіть програму, яка обчислює відносну частоту появи певних слів зі списку (які вводяться з командного рядочку) в даному тексті, незалежно від регістру.
\end{enumerate}

\subsection{Рядки -1}
\begin{enumerate}
\def\labelenumi{\arabic{enumi}.}

\item  Надрукувати заданий рядок:
\begin{enumerate}[label=\xslalph*)]
\item виключивши з нього всі цифри і подвоївши знаки '+' та '-';
\item виключивши з нього всі знаки '+', безпосередньо за якими знаходиться цифра;
\item виключивши з нього всі літери 'в', безпосередньо перед якими знаходиться літера 'с';
\item замінивши в ньому всі пари 'ph' на літеру 'f';
\item виключивши з нього всі зайві пропуски, тобто з кількох, що йдуть підряд, залишити один.
\end{enumerate}

\item
Уявіть, що ви працюєте у великій компанії, де використовується модульна архітектура. Ваш колега написав модуль із якоюсь логікою (ви не знаєте) і передає у вашу програму якісь дані. Ви пишете функцію яка зчитує дві змінних типу string, а повертає число типу int яке потрібно отримати додаванням цих рядків.

Але не було б все так просто, адже ваш колега не пише на Java, і він злий через те, що за java платять більше. Тому він вирішив пожартувати з вас і підсунув вам каверзу. Він придумав вставляти сміття в рядки перед тим, як викликати вашу функцію.

Тому попередньо вам потрібно прибрати з них сміття і конвертувати до числа. Під сміттям маються на увазі зайві символи та спеціалізовані знаки. 

Sample Input:

%^80 hhhhh20&&&&nd

Sample Output:

100 

\item
У звичних нам редакторах електронних таблиць є зручне уявлення числа з роздільником розрядів у вигляді пропуску, крім того у нас ціла частина від дробової відокремлюється комою. Набір таких чисел був експортований у формат CSV, де як роздільник використовується символ ";".

На стандартне введення ви отримуєте 2 таких дійсні числа, як результат потрібно вивести результат ділення першого числа на друге з точністю до чотирьох знаків після "крапки" - результат пітрібно вивести в звичайному вигляді. 

\item  Дано рядок, серед символів якого є принаймні одна кома, а може й немає її. Знайти номер

 \begin{enumerate}[label=\xslalph*)]
 \item першої по порядку коми;
 \item останньої по порядку коми;
 \item кількості ком.
 \end{enumerate}

\item  Виключити з заданого рядка групи символів, які знаходяться між '(' та ')'. Самі дужки теж мають бути виключені. Перевірте перед цим, що дужки розставлено правильно (парами) та всередині кожної пари дужок немає інших дужок.
\item  Заданий рядок, серед символів якого міститься двокрапка ':'. Отримати всі символи, розміщені
\begin{enumerate}[label=\xslalph*)]
\item до першої двокрапки включно;
\item після першої двокрапки;
\item між першою і другою двокрапкою. Якщо другої двокрапки немає, то отримати всі символи, розміщені після єдиної двокрапки.
\end{enumerate}
\item  Заданий текст надрукувати по рядках, розуміючи під рядком або наступні 6 символів, якщо серед них немає коми(оклику, питання), або частину тексту до коми включно.
\item  Задана послідовність символів, яка має вигляд:
  $d_1 \pm d_2 \pm \ldots \pm d_{n}$
($d_i$-цілі(дійсні) числа, $n>1$), за якою знаходиться крапка. Обчислити значення цієї алгебраїчної суми.
\item  Задане натуральне число n. Надрукувати в заданій системі числення b цілі числа від 0 до n.
\item  В заданий рядок входять тільки цифри та літери. Визначити, чи задовольняє він наступній властивості:
\begin{enumerate}[label=\xslalph*)]
\item
 рядок є десятковим записом числа, кратного 9 (6, 4);
\item рядок починається з деякої ненульової цифри, за якою знаходяться тільки літери і їх кількість дорівнює числовому значенню цієї цифри;
\item рядок містить (крім літер) тільки одну цифру, причому її числове значення дорівнює довжині рядка;
\item сума числових значень цифр, які входять в рядок, дорівнює довжині рядка;
\item рядок співпадає з початковим (кінцевим, будь-яким) відрізком ряду 0123456789;
\item рядок складається тільки з цифр, причому їх числові значення складають арифметичну прогресію (наприклад, 3 5 7 9, 8 5 2, 2).

\end{enumerate}
\item  Знайти у даному рядку символ та довжину найдовшої послідовності однакових символів, що йдуть підряд.
\item  Скласти  програму підрахунку загального числа входжень символів '+', '-', '*' у рядок А.
\item  Скласти  програму перетворення рядка А, замінивши у ньому всі знаки оклику '!' крапками '.', кожну крапку – трьома крапками '...', кожну зірочку '*'- знаком '+'.
\item  Рядок називається симетричним, якщо його символи, рівновіддалені від початку та кінця рядка, співпадають. Порожній рядок вважається симетричним. Перевірити рядок A на симетричність.
\item  Скласти програму видалення із рядка А всіх входжень заданої групи символів.
\item  Скласти  програму перетворення слова А, видаливши у ньому кожний символ '*' та подвоївши кожний символ, відмінний від '*'.
\item  Скласти  програму підрахунку найбільшої кількості цифр, що йдуть підряд, у рядку А.
\item  Скласти  програму підрахунку числа входжень у рядок А заданої послідовності літер.
\item  Скласти  програму, яка за рядком А та символом S будує новий рядок, отриманий заміною кожного символу, слідуючого за S, заданим символом С.
\item  Cкласти  програму перетворення рядка А видаленням із нього всіх ком, які передують першій крапці, та заміною у ньому знаком '+' усіх цифр '3', які зустрічаються після першої крапки.
\item  Cкласти  програму виведення на друк усіх цифр, які входять в заданий рядок, та окремо - решту символів, зберігаючи при цьому взаємне розташування символів у кожній з цих двох груп.
\item  Рядок називається монотонним, якщо він складається з зростаючої або спадної послідовності символів. Cкласти  програму перевірки монотонності рядка.
\item  Перевірити, чи складається рядок з
а) 2 симетричних підрядків;
б) n симетричних підрядків.
\item  Знайти символ, кількість входжень якого у рядок A
а) максимальна;
б) мінімальна.
\item  Дано рядок A, що містить послідовність слів. Скласти програми, що визначають:
\begin{enumerate}[label=\xslalph*)]
\item кількість усіх слів;
\item кількість слів, що починаються із заданого символа c;
\item кількість слів, що закінчуються заданим символом c;
\item кількість слів, що починаються й закінчуються заданим символом c;
\item кількість слів, що починаються й закінчуються однаковим символом.

\end{enumerate}
\item  Виділити з рядка A найбільший підрядок, перший і останній символи якого співпадають.
\item  Виділити з рядка найбільший монотонний підрядок, коди послідовних символів якого відрізняються на 1.
\item  Замінити всі пари однакових символів рядка, які йдуть підряд, одним символом. Наприклад, рядок ‘aabcbb’ перетворюється у ‘abcb’.
\item  Побудувати рядок S з рядків S1, S2 так, щоб у S входили
а) ті символи S1, які не входять у S2;
а) всі символи S1, які не входять у S2, та всі символи S2, які не входять у S1.
\item  Видалити з рядка симетричні початок та кінець. Наприклад, рядок ‘abcdefba’ перетворюється у ‘cdef’.
\item  Cкласти  програму виведення на друк тільки маленьких літер українського алфавіту, які входять в заданий рядок.
\item  Заданий рядок, який складається з великих літер українського алфавіту. Скласти  програму перевірки впорядкованості цих літер за алфавітом.
\item  Скласти  програму виведення на друк в алфавітному порядку усіх різних маленьких українських літер, які входять до даного рядка.
\item  Написати програму, яка виконує зсув по ключу (ключ задається) тільки для малих латинських літер. Наприклад: вхідні дані  anz – рядок, 2 – ключ. Результат: cpb. 


\item
Даний рядок. 
Необхідно визначити, чи є цей рядок IP адресою і якщо є вивести 4 числа в тому ж рядку.  
Формат виведення: YES/NO та числа адреси якщо рядок є IP адресою: 

Приклад 1: 127.0.0.1 - результат YES 127 0 0 1

Приклад 2: 256.0.0.1 - результат: NO

Приклад 3: LOL - результат: NO

\item
Визначте, чи може даний рядок бути валідною адресою публічної електронної пошти. 

Для логіну валідні: символи латиниці в малому та великому регістрі, символи: точка, нижнє підкреслення, тире, а також цифри. Довжина логіну не може перевищувати 32 символи.

Електронна пошта складається з імені користувача, символу @ та домену.

Валідним доменним ім'ям вважатимемо рядок виду ім'я компанії, точка, країна розміщення (univ.ua, toyoya.jp і т.д.)

Довжина найменування компанії використовуються символи з того ж набору, що і для логіну, довжина так само не повинна перевищувати 32 символи.

Країна розміщення: 2 символи в латиниці в малому регістрі, а також рядки com, net ,org
 
\item
Дано два рядки. Необхідно визначити скільки разів другий рядок зустрічається у першому.

Приклад 1: aabaaab aab  - Результат: 2

Приклад 2: aaaaaaaa aa  - Результат: 7

\item

Паліндром із двох рядків.  Дано два рядки. Якщо останній символ першого рядка відповідає першому символу другого рядка, такі символи можна видалити. Таку процедуру можна зробити скільки завгодно разів.

Чи можна за допомогою вищезгаданої операції отримати паліндром шляхом склеювання двох рядків, що залишилися? Вивести: YES / NO

Приклад 1: abc cba  - Результат: YES 
Приклад 2: 1234 664321  - Результат: YES 

\item
Паліндром із двох рядків V2. 

Дано два рядки. Дозволяється видалити символ з обох рядків, якщо:

1) Перша літера першого слова відповідає першій чи останній літері другого

2) Остання літера першого слова відповідає першій чи останній літері другого

Процедуру можна зробити скільки завгодно разів.

Чи можна за допомогою вищезгаданої операції отримати паліндром шляхом склеювання двох рядків, що залишилися?

Порожній рядок є паліндромом! Вивести: YES / NO

Приклад 1: 12345 14325  - Результат: YES 

Приклад 2: 123456789 239876541  - Результат: YES 

Приклад 3: 12345 12345  - Результат: YES 

\item
Кінцеві рухи.   Наведено список чисел. Дозволяється перекласти число з початку списку до кінця або з кінця до початку.
Чи можна відсортувати список за допомогою цих двох операцій?
Приклад 1: 1 2 2 4 5 1 2 2  - Результат: NO

Приклад 2: 1 2 3 4 5 0 1 1  - Результат: YES

Приклад 3: 5 6 7 1 2 3 8  - Результат: NO

\item
Категорії.  Рядок із змістом є - ім'я категорії : об'єкт1, об'єкт2, ...., об'єктН
У першому рядку дано кількість рядків у тексті із змістом N та кількість запитів до системи K,
Далі містяться слова K - запити в семантичне ядро ​​системи.

2 4

овочі: огірок, помідор, баклажан

автомобілі : кіа, шкода, мерседес, бмв

огірок кіа бмв ананас

Вивести потрібно:

овочі автомобілі автомобілі ХЗ

На кожен із запитів необхідно відповідати до якої категорії відноситься слово або ХЗ, якщо ядро ​​не знає категорії слова.

Приклад:

2 4

овочі: огірок, помідор, баклажан

автомобілі : кіа, шкода, мерседес, бмв

огірок кіа бмв ананас

Результат:

овочі автомобілі автомобілі ХЗ 

\item

\end{enumerate}

\subsection{Рядки -2 (regular expressons)}
\begin{enumerate}
\def\labelenumi{\arabic{enumi}.}
 
\item  Дано рядок. Групи символів, що відокремлені пропусками (одним  або  кількома)  і  не містять  пропусків  усередині,  називатимемо словами. Скласти підпрограми для: 
\begin{enumerate}[label=\xslalph*)]
\item знаходження найдовшого слова; 
\item визначення кількості слів 
\item вилучення з рядку зайвих пропусків і всіх слів, що складаються з однієї літери; 
\item вилучення всіх пропусків на початку рядків, у кінці рядків і між словами (крім одного); 
\item вставки пропусків до рядків рівномірно між словами так, щоб довжина всіх рядків (якщо в них більше 1 слова) була 80 символів і кількість пропусків між  словами  в  одному  рядку  відрізнялась  не  більш  ніж  на  1 
(вважати, що рядки файла мають не більш ніж 80 символів). 
\end{enumerate}

Зробити 2 варіанти: 

1) Результат записати в новий рядок.

2) Результат помістити в цей самий рядок.


\item  В заданий рядок входять тільки цифри та літери. Перевірте це. Визначити, чи задовольняє він наступній властивості:
\begin{enumerate}[label=\xslalph*)]
\item рядок є десятковим записом числа, кратного 9 (6, 4);
\item рядок починається з деякої ненульової цифри, за якою знаходяться тільки літери і їх кількість дорівнює числовому значенню цієї цифри;
\item рядок містить (крім літер) тільки одну цифру, причому її числове значення дорівнює довжині рядка;
\item сума числових значень цифр, які входять в рядок, дорівнює довжині рядка;
\item рядок співпадає з початковим (кінцевим, будь-яким) відрізком ряду 0123456789;
\item рядок складається тільки з цифр, причому їх числові значення складають арифметичну прогресію (наприклад, 3 5 7 9, 8 5 2, 2).




\end{enumerate}
\item  Знайти у даному рядку символ та довжину найдовшої послідовності однакових символів, що йдуть підряд.

\item   Скласти  програму підрахунку загального числа входжень символів '+', '-', '*' у рядок А.
\item   Скласти  програму перетворення рядка А, замінивши у ньому всі знаки оклику '!' крапками '.', кожну крапку – трьома крапками '...', кожну зірочку '*'- знаком '+'.

\item  Інверсія рядка A - це рядок B, записаний тими ж символами у зворотньому порядку. Інверсія порожнього рядка за означенням – порожній рядок. Побудувати інверсію рядка. Результат в цьому самому рядку.
\item   Скласти програму видалення із рядка А всіх входжень заданої групи символів.
\item   Написати програму, яка виконує зсув по ключі (ключ задається) тільки для малих латинських та українських літер. Наприклад: вхідні дані  anz – рядок, 2 – ключ. Результат: cpb.
\item  Знайти символ, кількість входжень якого у рядок A
а) максимальна;
б) мінімальна.

\item  Визначити процедуру пошуку в рядку підрядків, фрагментом яких є заданий регулярний вираз. 

\item   Знайти у даному рядку символ та довжину найдовшої послідовності однакових символів, що йдуть підряд та видалити їх.

\item  Виділити з рядка найбільший монотонний підрядок, коди послідовних символів якого відрізняються на 1.

\item   Замінити всі пари однакових символів рядка, які йдуть підряд, одним символом. Наприклад, рядок ‘aabcbb’ перетворюється у ‘abcb’.


\item  
Дано рядок A, що містить послідовність слів. Скласти програми, що визначають:
\begin{enumerate}[label=\xslalph*)]
\item  кількість усіх слів;
\item кількість слів, що починаються із цифри та вивести їх
\item кількість слів, що закінчуються з тризначного числа та вивести їх всіх
\item кількість слів, що починаються й закінчуються одним символом
\item кількість слів, що є електронною поштою
\end{enumerate}

\item  
Перевірити, чи складається рядок з
\begin{enumerate}[label=\xslalph*)]
\item 2 симетричних слів;
\item n симетричних слів.
\item римських чисел
\item чи є він записом поштової адреси (email)
\end{enumerate}

\item   Cкласти  програму виведення на друк тільки маленьких літер українського алфавіту, які входять в заданий рядок.

\item  Заданий рядок, який складається з слів, ідентифікаторів та 7-значних номерів телефону(з рисками та без всередині). Скласти  програму що впорядковє ці слова та виводить не екран в цьому порядку

\item   Перевірити що даний рядок є коректним рядком програми на мові Java. (ключові слова обмежимо простими типами та умовними операторами)

\item   Перевірити що даний рядок є коректним арифметичним виразом

\item   Знайти всі дати за американським стандартом та замінити їх на український запис дати
\end{enumerate}

\subsection{Текстові файли }
\begin{enumerate}
\def\labelenumi{\arabic{enumi}.}

\item  Дано текстовий файл. Групи символів, що відокремлені пропусками (одним  або  кількома)  і  не містять  пропусків  усередині,  називатимемо словами. Скласти підпрограми для: 
\begin{enumerate}[label=\xslalph*)]
\item знаходження найдовшого слова у файлі; 
\item визначення кількості слів у файлі; 
\item вилучення з файла зайвих пропусків і всіх слів, що складаються з 
однієї літери; 
\item вилучення всіх пропусків на початку рядків, у кінці рядків і між сло-
вами (крім одного); 
\item вставки пропусків до рядків рівномірно між словами так, щоб дов-
жина всіх рядків (якщо в них більше 1 слова) була 80 символів і кількість 
пропусків між  словами  в  одному  рядку  відрізнялась  не  більш  ніж  на  1 
(вважати, що рядки файла мають не більш ніж 80 символів). 
\end{enumerate}
Результат записати до файла H.

\item  Визначити функцію, яка:
 \begin{enumerate}[label=\xslalph*)]
\item підраховує кількість порожніх рядків; 
\item обчислює максимальну довжину рядків текстового файла. 
\end{enumerate}
\item  Визначити процедуру виведення: 
\begin{enumerate}[label=\xslalph*)]
\item усіх рядків текстового файла; 
\item рядків, які містять більше 60 символів. 

\item  Визначити функцію, що визначає кількість рядків текстового файла, що: 
\item починаються із заданого символу; 
\item закінчуються заданим символом; 
\item починаються й закінчуються одним і тим самим символом; 
\item складаються з однакових символів.
\end{enumerate}

\item  Визначити процедуру, яка переписує до текстового файла G усі 
рядки текстового файла F: 
\begin{enumerate}[label=\xslalph*)]
\item із заміною в них символа '0' на '1', і навпаки; 
\item в інвертованому вигляді. 
\end{enumerate}

\item  Визначити  процедуру  пошуку  найдовшого  рядка  в  текстовому 
файлі. Якщо таких рядків кілька, знайти перший із них. 


\item  Визначити  процедуру,  яка  переписує  компоненти  текстового 
файла F до файла G, вставляючи до початку кожного рядка один сим-
вол пропуску. Порядок компонент не має змінюватися.

\item  У текстовому файлі записано непорожню послідовність дійсни
чисел,  які  розділяються  пропусками.  Визначити  функцію  обчисленн
найбільшого з цих чисел. 

\item  У  текстовому файлі F  записано  послідовність  цілих  чисел,  як
розділяються  пропусками.  Визначити  процедуру  запису  до  текстовог
файла g усіх додатних чисел із F. 

\item  У  текстовому файлі  кожний  рядок  містить  кілька  натуральни
чисел, які розділяються пропусками. Числа визначають вигляд  геомет
ричної фігури (номер) та її розміри. Прийнято такі домовленості: 
 відрізок прямої задається координатами своїх кінців і має номер 1;
 прямокутник задається координатами верхнього лівого й нижнього
правого кутів і має номер 2; 
 коло задається координатами центра й радіусом і має номер 3. 
Визначити процедури обчислення: 
\begin{enumerate}[label=\xslalph*)]
\item відрізка з найбільшою довжиною; 
\item прямокутника з найбільшим периметром; 
\item кола з найменшою площею.
\end{enumerate}

\item  Відомості про учня складаються з його імені, прізвища та назви 
класу (рік  навчання  та  літери),  в  якому  він  вчиться.  Дано файл,  який 
містить відомості про учнів школи. Скласти підпрограми, які дозволяють:
\begin{enumerate}[label=\xslalph*)] 
\item визначити, чи є в школі учні з однаковим прізвищем; 
\item визначити, чи є учні з однаковим прізвищем у паралельних класах; 
\item визначити, чи є учні з однаковим прізвищем у певному класі; 
\item відповісти на питання а)-в) стосовно учнів, у яких збігаються ім'я та 
прізвище; 
\item визначити, в яких класах налічується більше 35 учнів; 
\item визначити, на скільки учнів у восьмих класах більше, ніж у десятих; 
\item  зібрати  у файл  відомості  про  учнів  9-10-х  класів,  розташувавши 
спочатку відомості про учнів класу 9 а, потім – 9 б тощо; 
\item отримати список учнів даного класу за зразками: 
  Прізвище Ім'я 
  Прізвище І. 
  І.Прізвище. 
\end{enumerate}

\item  Дано файл, який містить ті самі відомості про учнів школи, що й 
в  попередній  задачі,  і  додатково  оцінки,  отримані  учнями  на  іспитах  із 
заданих предметів. Скласти процедури для:
\begin{enumerate}[label=\xslalph*)]
\item
 визначення кількості учнів, які не мають оцінок, нижче 4;  
\item побудови файла, який містить відомості про кращих учнів ш
що мають оцінки, не нижче 4; 
\item друкування відомостей про учнів, які мають принаймні одну 
довільну оцінку, у вигляді прізвища та ініціалів, назви класу, предме
\end{enumerate}

\item  Відомості про автомобіль складаються  з його марки, номе
прізвища власника. Дано файл, який містить відомості про кілька 
мобілів. Скласти процедури знаходження: 
\begin{enumerate}[label=\xslalph*)]
\item
 прізвищ власників номерів автомобілів певної марки; 
\item кількості автомобілів кожної марки.
\end{enumerate}
\item  Дано файл,  який  містить  відомості  про  книжки.  Відомості  пр
кожну книгу – це прізвище автора, назва та рік видання. Скласти проц
дури пошуку: 
\begin{enumerate}[label=\xslalph*)]
\item назв книг певного автора, виданих із 1960 р.; 
\item книг  із заданою назвою. Якщо така книжка є, то надрукувати пр
звища авторів і рік видання. 
\end{enumerate}

\item  Дано файл, який містить номери телефонів співробітників уст
нови:  вказуються  прізвище  співробітника,  його  ініціали  та  номер  тел
фону.  Визначити  процедуру  пошуку  телефону  співробітника  за  йо
прізвищем та ініціалами. 


\item  Дано файл з відомостями про кубики: розмір кожного (довжин
ребра  у  см),  його  колір (червоний,  жовтий,  зелений,  синій)  і  матеріа
(дерев'яний, металевий, картонний). Скласти процедури пошуку: 
\begin{enumerate}[label=\xslalph*)]
\item кількості кубиків кожного з перелічених кольорів, їх сумарний об'єм
\item кількості дерев'яних кубиків  із ребром 3 см  і металевих кубиків 
ребром, більшим за 5 см.
\end{enumerate}
\end{enumerate}

\section{6. Введення-виведення. Бінарні файли. Серіалізація}
 \subsection{Бінарні файли}

\begin{enumerate}
\def\labelenumi{\arabic{enumi}.}
\item  Робота з файлом з цілих чисел. Створити бінарний файл з випадкових цілих чисел.
\begin{enumerate}[label=\xslalph*)]
\item   Прочитати з бінарного файлу цілі значення (до кінця файлу), знайти добуток парних елементів та вивести в інший бінарний файл. 
\item   Прочитати з бінарного файлу цілі значення (до кінця файлу), замінити від'ємні значення модулями, додатні нулями та вивести отримані значення в інший бінарний файл. 
\item   Прочитати з бінарного файлу цілі значення (до кінця файлу), замінити розділити парні елементи на 2, непарні – збільшити у 2 рази та вивести отримані значення в інший бінарний файл.
 \end{enumerate}


\item  Робота з файлом з дійсних чисел. Вввести з консолі дійсні числа та записати їх в файл.
\begin{enumerate}[label=\xslalph*)]
\item
  Прочитати з бінарного файлу дійсні значення (до кінця файлу), знайти їх суму та вивести на консоль. 
\item   Прочитати з бінарного файлу дійсні значення (до кінця файлу), знайти добуток модулів ненульових елементів та вивести в інший бінарний файл. 
 \end{enumerate}

\item

Створить клас Студент - який містить інформацію про студента (ПІБ, курс, номер заліковки, кількість зданих предметів та
відповідний масив оцінок). Реалізуйте метод, що дозволяє створювати файл з даного масиву студентів та метод для
додавання запису про студента в масив.
Реалізуйте метод, що знаходить студента з даним прізвищем у даному файлі та виводить його в інший файл та метод 
знаходження прізвища студента з найгіршим середнім балом.  
 
\item
Створити клас для роботи з комплексними числами та написати клас, що дозволяє серіалізувати/десеріалізувати обєкти комплексних чисел.

\end{enumerate}
 \subsection{Самостійна робота}
\begin{enumerate}
\def\labelenumi{\arabic{enumi}.}
\item   Дано файл, компоненти якого є дійсними числами. Скласти підпрограми для обчислення: 
\begin{enumerate}[label=\xslalph*)]
\item
 суми компонент файла;  
\item кількості від'ємних компонент файла;  
\item останньої компоненти файла; 
\item найбільшого зі значень компонент файла; 
\item найменшого зі значень компонент файла з парними номерами;
\item суми найбільшого та найменшого зі компонент; 
\item різниці першої й останньої компоненти файла; 
\item  кількості  компонент файла,  менші  за  середнє  арифметичне  всі
його компонент. 
\end{enumerate}
\item   Дано файл, компоненти якого є цілими числами. Скласти підпрограми для обчислення: 
\begin{enumerate}[label=\xslalph*)]
\item кількості парних чисел серед компонент; 
\item кількості квадратів непарних чисел серед компонент; 
\item різниці між найбільшим парним і найменшим непарним числами 
компонент; 
\item  кількості  компонент  у найдовшій  зростаючій послідовності  компо
нент файла.
\end{enumerate}
\item   Дано файл F,  компоненти  якого  є  цілими  числами. Побудувати 
файл G, який містив би всі компоненти файла F: 
\begin{enumerate}[label=\xslalph*)]
\item що є парними числами;   \item що діляться на 3 і на 5; 
\item що є точними квадратами;        \item записані у зворотному порядку; 
\item за винятком повторних входжень одного й того самого числа. 
\end{enumerate}
\item   Використовуючи файл F,  компоненти  якого  є  цілими  числами, 
побудувати файл G, що містить усі парні числа файла F, і файл H – усі 
непарні. Послідовність чисел зберігається. 

\item  Задано натуральне число n  та файл F,  компоненти якого є ці-
лими  числами.  Побудувати  файл  G,  записавши  до  нього  найбільше 
значення перших n компонент файла F, потім – наступних n компонент 
тощо. Розглянути два випадки:
\begin{enumerate}[label=\xslalph*)]
\item кількість компонент файла ділиться на n; 
\item кількість компонент файла не ділиться на n. Остання компонента 
файла  g має  дорівнювати  найбільшій  із  компонент файла F,  які  утво-
рюють останню (неповну) групу. 
\end{enumerate}
\item  Дано файл F, компоненти якого є цілими числами. Файл містить 
рівне число додатних і від'ємних чисел. Використовуючи допоміжний файл 
H, переписати компоненти файла F до файла G так, щоб у файлі G: 
\begin{enumerate}[label=\xslalph*)]
\item не було двох сусідніх чисел одного знака; 
\item спочатку йшли додатні, потім – від'ємні числа; 
\item числа йшли таким чином: два додатних, два від'ємних тощо (при-
пускається, що число компонент у файлі F ділиться на 4).
\end{enumerate}
\item  Дано файл,  який  містить  відомості  про  прямокутники:  указано 
номер прямокутника у файлі, координати верхнього лівого кута, нижньо-
го правого кута прямокутника. Скласти процедуру пошуку прямокутника 
з найбільшою площею й визначення цієї площі.
\item
 У двох файлах міститься  таблиця футбольного  турніру, у пер-
шому – записано назви команд; у другому – результати матчів, що збе-
рігаються у записах типу T\_Match 
{ 
  int n1, n2;    int b1, b2; 
} T\_Match; 
Тут  у  структурі  типу  T\_Match  поля  n1,  n2 –  номери  першої  і  другої 
команд (тобто номери назв команд у файлі команд); b1, b2 – кількість 
м'ячів, забитих першою та другою командами, відповідно. 
Кожній  команді  за перемогу нараховується 3 очки,  за  нічию – 1,  за 
поразку – 0. 
Із двох команд, які мають однакову кількість очок, першою вважаєть-
ся та, що 
\begin{itemize}
\item   має кращу різницю забитих і пропущених м'ячів; 
\item   за однакової різниці має більше забитих м'ячів; 
\item   за всіма однаковими попередніми показниками визначається жеребкуванням 
(для жеребкування використати генератор випадкових чисел). 
\end{itemize}
Знайти команду, яка є лідером. 
Указівка.  Описати  підпрограми  створення  файлів  команд  і  матчів, 
додавання результату матчу, визначення лідера.
\end{enumerate}

\subsection{Серіалізація}

На стандартный ввод подаются данные о студентах университетской группы в формате JSON:

{
    "ID":134,
    "Number":"ИЛМ-1274",
    "Year":2,
    "Students":[
        {
            "LastName":"Вещий",
            "FirstName":"Лифон",
            "MiddleName":"Вениаминович",
            "Birthday":"4апреля1970года",
            "Address":"632432,г.Тобольск,ул.Киевская,дом6,квартира23",
            "Phone":"+7(948)709-47-24",
            "Rating":[1,2,3]
        },
        {
            // ...
        }
    ]
}

В сведениях о каждом студенте содержится информация о полученных им оценках (Rating). Требуется прочитать данные, и рассчитать среднее количество оценок, полученное студентами группы. Ответ на задачу требуется записать на стандартный вывод в формате JSON в следующей форме:

{
    "Average": 14.1
}


\begin{enumerate}
\def\labelenumi{\arabic{enumi}.}

\item  Серіалізація: Багаж пасажира характеризується номером пасажира, кількістю
речей  і  їхньою  загальною  вагою.  Дано  файл  пасажирів,  який  містит
прізвища пасажирів,  і файл, що містить  інформацію  про  багаж  кілько
пасажирів (номер  пасажира –  це  номер  запису  у  файлі  пасажирів)
Скласти процедури для: 
\begin{enumerate}[label=\xslalph*)]
\item  знаходження  пасажира,  у  багажі  якого  середня  вага  однієї  реч
відрізняється не більш ніж на 1 кг від загальної середньої ваги речей; 
\item визначення пасажирів, які мають більше двох речей,  і пасажирів
кількість речей у яких більша за середню кількість речей; 
\item видачі відомостей про пасажира, кількість речей у багажі якого н
менша, ніж у будь-якому іншому багажі, а вага речей – не більша, ніж 
будь-якому іншому багажі із цією самою кількістю речей; 
\item визначення, чи мають принаймні два пасажири багажі, які не відр
зняються за кількістю речей  і відрізняються вагою не більш ніж на 1 к
(якщо такі пасажири є, то показати їхні прізвища); 
\item визначення пасажира, багаж якого складається з однієї речі вагою
не менше 30 кг.
\end{enumerate}
\item  Серіалізація: Дано файл,  який містить  відомості  про  іграшки:  указано  назву 
іграшки (напр., м'яч,  лялька,  конструктор  тощо),  її  вартість  у  гривнях  і 
вікові межі для дітей, яким  іграшка призначається (напр., для дітей від 
двох до п'яти років). Скласти процедури: 
\begin{enumerate}[label=\xslalph*)]
\item пошуку назв  іграшок, вартість яких не перевищує 40 грн, призна-
чених дітям п'яти років; 
\item пошуку назв іграшок, призначені дітям і чотирьох, і десяти років; 
\item пошуку назв найдорожчих іграшок (ціна яких відрізняється від ціни 
найдорожчої іграшки не більш ніж на 50 грн); 
\item визначення ціни найдорожчого конструктора; 
\item визначення ціни всіх кубиків; 
\item пошуку  двох  іграшок,  що  призначені  дітям  трьох  років,  сумарна 
вартість яких не перевищує 20 грн; 
\item пошуку конструктора ціною 22 грн, призначеного дітям від п'яти до 
десяти років. Якщо такої іграшки немає, то занести відомості про  її від-
сутність до файла. 
\end{enumerate}

\item  Нехай множина цілих чисел задана у файлі. Визначити: 
\begin{enumerate}[label=\xslalph*)]
\item процедуру введення множини; 
\item процедуру виведення множини; 
\item процедуру доповнення множини; 
\item процедуру видалення елемента з множини; 
\item функцію, що дає відповідь, чи входить елемент до множини; 
\item функцію, що дає відповідь, чи порожня множина; 
\item функцію, що знаходить максимальний елемент множини; 
\item функцію, що знаходить мінімальний елемент множини; 
\item процедуру об'єднання множин; 
\item процедуру різниці множин; 
\item процедуру перетину множин; 
\item функцію обчислення ваги множини; 
\item функцію обчислення діаметра множини;
 \end{enumerate}

\item  Дано файл,  компоненти  якого  є  записи (koef, st) –  коефіцієнт  і 
степінь членів полінома ($koef \neq 0$). Визначити підпрограми для виконан-
ня таких дій над поліномом:
\begin{enumerate}[label=\xslalph*)]
\item введення полінома; \item друк полінома; 
\item обчислення похідної від полінома; 
\item обчислення невизначеного інтеграла від полінома; 
\item упорядкування за степенями елементів полінома; 
\item приведення подібних серед елементів полінома; 
\item додавання, віднімання двох поліномів; 
\item множення двох поліномів; 
\item знаходження степеня полінома; 
\item з'ясування, чи має поліном корені, рівні нулю,  і визначення  їхньої кратності; 
\end{enumerate}
\end{enumerate}

\subsection{Серіалізація}
\begin{enumerate}
\def\labelenumi{\arabic{enumi}.}
\item   Описати класи Факультет та Інститут (з полем – масивом факультетів). Створити об'єкти, здійснити їх бінарну серіалізацію й десеріалізацію. 
\item   Створити схему документу та XML-документ, який описує дані про користувача. Згенерувати класи за допомогою технології JAXB. 
\item   Створити схему документу та XML-документ, який описує дані про книгу. Згенерувати класи за допомогою технології JAXB. 
\item   Створити схему документу та XML-документ, який описує дані про місто. Згенерувати класи за допомогою технології JAXB. 
\item   Створити схему документу та XML-документ, який описує дані про кінофільм. Згенерувати класи за допомогою технології JAXB. 
\item   Описати класи Факультет та Інститут (з полем – масивом факультетів). Створити об'єкти, здійснити їх серіалізацію й десеріалізацію в XML. 
\end{enumerate}


\end{document}


