\documentclass[]{article}
\usepackage{lmodern}
\usepackage{amssymb,amsmath}
\usepackage{ifxetex,ifluatex}


\usepackage[utf8]{inputenc}
\usepackage[english,russian,ukrainian]{babel}

\usepackage{fixltx2e} % provides \textsubscript
\ifnum 0\ifxetex 1\fi\ifluatex 1\fi=0 % if pdftex
  \usepackage[T1]{fontenc}
  \usepackage[utf8]{inputenc}
\else % if luatex or xelatex
  \ifxetex
    \usepackage{mathspec}
  \else
    \usepackage{fontspec}
  \fi
  \defaultfontfeatures{Ligatures=TeX,Scale=MatchLowercase}
\fi
% use upquote if available, for straight quotes in verbatim environments
\IfFileExists{upquote.sty}{\usepackage{upquote}}{}
% use microtype if available
\IfFileExists{microtype.sty}{%
\usepackage{microtype}
\UseMicrotypeSet[protrusion]{basicmath} % disable protrusion for tt fonts
}{}
\usepackage[unicode=true]{hyperref}
\hypersetup{
            pdfborder={0 0 0},
            breaklinks=true}
\urlstyle{same}  % don't use monospace font for urls
\usepackage{graphicx,grffile}
\makeatletter
\def\maxwidth{\ifdim\Gin@nat@width>\linewidth\linewidth\else\Gin@nat@width\fi}
\def\maxheight{\ifdim\Gin@nat@height>\textheight\textheight\else\Gin@nat@height\fi}
\makeatother
% Scale images if necessary, so that they will not overflow the page
% margins by default, and it is still possible to overwrite the defaults
% using explicit options in \includegraphics[width, height, ...]{}
\setkeys{Gin}{width=\maxwidth,height=\maxheight,keepaspectratio}
\IfFileExists{parskip.sty}{%
\usepackage{parskip}
}{% else
\setlength{\parindent}{0pt}
\setlength{\parskip}{6pt plus 2pt minus 1pt}
}
\setlength{\emergencystretch}{3em}  % prevent overfull lines
\providecommand{\tightlist}{%
  \setlength{\itemsep}{0pt}\setlength{\parskip}{0pt}}
\setcounter{secnumdepth}{0}
% Redefines (sub)paragraphs to behave more like sections
\ifx\paragraph\undefined\else
\let\oldparagraph\paragraph
\renewcommand{\paragraph}[1]{\oldparagraph{#1}\mbox{}}
\fi
\ifx\subparagraph\undefined\else
\let\oldsubparagraph\subparagraph
\renewcommand{\subparagraph}[1]{\oldsubparagraph{#1}\mbox{}}
\fi

\date{}

\usepackage{multicol}

\usepackage{enumitem}
\makeatletter
\newcommand{\xslalph}[1]{\expandafter\@xslalph\csname c@#1\endcsname}
\newcommand{\@xslalph}[1]{%
    \ifcase#1\or а\or б\or в\or г\or д\or e\or є\or ж\or з\or i%
    \or й\or к\or л\or м\or н\or о\or п\or р\or с\or т%
    \or у\or ф\or х\or ц\or ч\or ш\or ю\or я\or аа\or бб\or вв %
    \else\@ctrerr\fi%
}
\AddEnumerateCounter{\xslalph}{\@xslalph}{m}
\makeatother


\begin{document}


Збірник задач для вивчення мови Java


\section{1. Простий клас Java. Командний рядок. Методи та функції}

\subsection{Аудиторна робота}
\setcounter{subsection}{1}

\begin{enumerate}
\def\labelenumi{1.\arabic{enumi}.}
\item
Створіть клас, що містить неініціалізовані поля типів int, char та Рядок, і надрукуйте їх значення, щоб переконатися, що Java виконує ініціалізацію за замовчуванням.
\item
Напишить програму “hello, world”. 
\item
Скомпілюйте програму з javac і запустіть її за допомогою java. Якщо ви використовуєте інше середовище розробки, ніж JDK, дізнайтеся, як компілювати та запускати програми в цьому середовищі.
 \item
Напишіть програму, яка демонструє, що незалежно від того, скільки об’єктів ви створюєте для певного класу, у цьому класі є лише один екземпляр певного статичного поля.

Командний рядок
\item
Напишіть програму, яка друкує три аргументи, взяті з командного рядка.
\item
Знайти суму аргументів командного рядка та порахувати введені дійсні числа.
\item
Ввести ім'я користувача за допомогою командного рядка та виведіть його в консоль.

\item
Задокументуйте правильно програму. Виконайте Javadoc для файлу та перегляньте результати за допомогою веб -браузера.
Перевірте отриману документацію за допомогою веб -браузера.
\item
 Додайте список елементів HTML до документації.
Візьміть програму попередбної вправи і додайте до неї документацію з коментарями. Розпакуйте цю документацію з коментарями у файл HTML за допомогою Javadoc і перегляньте її у веб -браузері.
\end{enumerate}

В приведених завданнях потрібно зчитати дані з консолі та вивести відповідь в консоль. В консолі також вивести прізвище студенту, час отримання завдания тай його номер. Додати коментар на початку в програми у вигляді /**коментар*/ та згенерувати html-файл документації. 

\begin{enumerate}
\def\labelenumi{2.\arabic{enumi}.}
\item
Вивести в косоль підказку «Введіть прізвище», введіть його з консолі тв виведіть повідомлення «Привіт, ***» де замість зірочок — введене прізвище.
\item
 Введіть два цілих числа та виведіть кількість їх розрядів (розвяжить це за допомогою рядків та використовуючі логарифм). Обчисліть середнє гармонічне цих чисел та виведіть його з точністю до двох знаків після коми.

\item Ввести два дійсних числа записаних через пробіли та обчислити значення їх середнього геометричного. Результат вивести в науковому та десятковому представленні. 

\item Ввести дійсне число градусів Цельсія $C$ (на екрані повинна бути підказка, що ввести) та обчислити й вивести число F в дійсному форматі – та сама температура в градусах Фаренгейта за формулою $F=9*C/5+32$. Результат вивести в окремому рядку вигляду «F=*** », де замість зірок представлення в найкоротшому вигляді з можливих.

\item Запустити програму з командного рядку. Відобразити в консолі аргументи командного рядку в зворотньому порядку.

\item Вивести задану кількість випадкових чисел з переходом і без переходу на новий рядок.

\item Ввести цілі числа як аргументи командного рядку, підрахувати їх добуток та вивести результат на консоль.

\end{enumerate}

\subsection{Прості типи та оператори}

\begin{enumerate}
\def\labelenumi{3.\arabic{enumi}.}
\item
Обчисліть наступні математичні вирази та виведіть результати:
1.2+31; 45*54-11; 15/4; 15.0/4; 67\%5; (2*45.1 +3.2)/2;


\item
Обчислить результати наступних виразів та вивести на екран напис
українською мовою «Результат дорівнює:»:

2+3; 4.5*56; 2/3.0.
\item
Виведіть напис: «Введить ім``я:»

Введіть з нового рядка ваше ім'я (наприклад, «Вася» ) та виведіть
привітання вигляду «Привіт, Вася!»

\item
Ініціалізуйте наступні числа як дійсні та подвійні дійсні:
\(10^{- 4}\), 2.33E5, \(\pi\) , \(e\), \( \sqrt{5}\), \(ln(100)\)
\item
Задайте в програмі довільні 5 цілих та 5 дійсних чисел. Вивести на екран таблицю
з цих значень у вигляді, слідкуючі за "красою" виводу:
\begin{verbatim}
x | 1 | 2 | 3 | 4 | 5 |
- | - - | - - | - - | - - | - - |
y | 3.0 | 1.0 | 5.0 | 4.0 | 2.1 |
\end{verbatim}

\item
Ініціалізувати два довільні рядки та вивести їх в одному рядку та
поставивши між ними кому та пробіл, а перед та після три окличних знаки.
Приклад:
\begin{verbatim}
!!! Hello , World! !!!
\end{verbatim}

\item
Обчислити силу притягання $F$ в науковому (екоспоненційному) форматі між двома тілами,
  що мають маси $m_{1},m_{2}$ на відстані $r$.
  \emph{\emph{Вказівка}}. Шукана силa визначається за формулою
  $ F=\gamma \frac{m_{1}*m_{2}}{r^{2}}$,
  де $\gamma = 6.673*10^{-11}$ Н*м\textsuperscript{2}/кг\textsuperscript{2}. Всі потрібні змінні
  присвоюються всередині програми. Результат вивести в окремому рядку
  вигляду «F=*** », де замість зірок представлення в науковому
  (експоненційному) вигляді.
\item
Ввести дійсне число $x$ та підрахуйте без та за допомогою математичних
функцій її цілу та дробову частину, найменше ціле число, що більше $x$
та найбільше ціле, що менше $x$, а також його округлене значення.
Перевірте результат роботи для від'ємного числа.
\item
Ввести в двох різних рядках послідовно два дійсних числа та обчислити
значення їх різниці та добутку. Результат вивести в десятковому 
представленні (з фіксованою крапкою) та науковому представленні.
\item
Ввести два дійсних числа записаних через пробіли в одному рядку та
обчислити значення їх середнього арифметичного та середнього
гармонічного. Результат вивести в науковому та десятковому
представленні.

\item
Ввести дійсне число від 0 до 10000 та вивести його 8 ступінь з точністю
до 20 знаків до десяткової коми та 4 значками після десяткової коми.

\item
Ввести ціле число та вивести його 8 ступінь якщо результат буде коректним 
для типу Integer. В іншому випадку вивести повідомлення "Переповнення типу" .

\item 
Ввести два цілих числа то обчислити їх добуток (так щоб він був коректний навіть при максимально великих цілих числах) 
та частку як дійсне число з максимально можливою кількістю цифр після десяткової крапки

\item
Ввести користуючись лише однією функцією вводу ціле число записане в
шістнадцятковому вигляді та вивести його зменшене на одиницю в
шістнадцятковому та десятковому вигляді.
\item
Дійсне число записано в рядку, при цьому перед ним може стояти будь-яка
послідовність з пробілів та символів `*'. Ввести його та виведіть значення його кубу.

\item
Ввести дійсне число х та обчислити значення функції тригонометричного
косинуса для нього.
\item
Обчислити гіпотенузу c прямокутного трикутника за катетами a та b.
\item
Обчислити площу трикутника S за трьома сторонами a, b, c.

\item
  В трикутнику відомо довжини всіх сторін. Обчислити довжини його:
  \begin{enumerate}[label=\xslalph*)]
   \item
    медіан,
   \item
    бісектрис,
    \item
    висот.
  \end{enumerate}
\item
Трикутник заданий величинами своїх кутів та радіусом вписаного кола.
Обчисліть його площу.
\item
Трикутник заданий довжиною своїх сторін. Знайти та вивести величину
кутів трикутника у радіанах та градусах.
\item
 Обчислити відстань від точки \((x_{0},y_{0})\) до:
\begin{enumerate}[label=\xslalph*)]
\item заданої точки \((x,y)\);
\item заданої прямої \(ax + by + c = 0\);
\item точки перетину прямих \(x + by + c = 0\) і
\(ax + y + c = 0,\ \) де 
\(ab \neq 1\).
\end{enumerate}
\item
Знайти об'єм циліндра, якщо відомо його радіус основи та висоту.
\item
Знайти об'єм конуса, якщо відомо його радіус основи та висоту.
\item
Знайти об'єм тора з внутрішнім радіусом $r$ і зовнішнім радіусом $R$.
\item
Знайти дійсні корені квадратного рівняння з коефіцієнтами a,b,c, якщо відомо,
що обидва корені в ньому існують. Перевірте ваш розв'язок на
коефіцієнтах рівняння a=3,b=100,c=2.
\end{enumerate}

\subsection{Декларація та використання статичних методів}

Під функціями в цьому розділі розуміються статичні методи класу.

\begin{enumerate}
\def\labelenumi{4.\arabic{enumi}.}
\item

 Напишіть функцію, яка за найменшу кількість арифметичних операцій,
  обчислює значення многочлена для введеного з клавіатури значення
  $x$:
  \begin{enumerate}[label=\xslalph*)]
  \item \(y = x^{4} + 2x^{2} + 1\); 
  \item \(y = x^{4} + x^{3} + x^{2} + x + 1\);
  \item \(y = x^{5} + 5x^{4} + 10x^{3} + 10x^{2} + 5x + 1\);
  \item \(y = x^{9} + x^{3} + 1\);
  \item \(y = 16x^{4} + 8x^{3} + 4x^{2} + 2x + 1\); 
  \item \(y = x^{5} + x^{3} + x\).
  \end{enumerate}
\item
 Скласти функцію для обчислення значення многочлена від двох змінних
  для введеної з клавіатури пари чисел \((x,y)\):
  \begin{enumerate}[label=\xslalph*)]
    \item
    \(f(x,y) = x^{3} + 3x^{2}y + 3xy^{2} + y^{3};\)
    \item
    \(f(x,y) = x^{2}y^{2} + x^{3}y^{3} + x^{4}y^{4};\)
    \item
    \(f(x,y) = x + y + x^{2} + y^{2} + x^{3} + y^{3} + x^{4} + y^{4}\).
  \end{enumerate}
\item
  Напишіть функцію $ Rosenbrock2d(x,y) = 100(x^{2} - y)^{2} + (x - 1)^{2}$ 
 та перевірте її результат на довільних трьох парах дійсних чисел.
\item
Трикутник заданий довжинами своїх сторін. Знайти периметр та площу цього
трикутника. Перевірте для значень сторін
\(a = 3,b = c = 3.5 + 3*2^{- 111}\)
\item
Трикутник вводиться координатами своїх вершин, які вводяться так: в
першому рядку через пробіл два дійсних числа --- координати точки А,
пропускається рядок, в третьому рядку через пробіл два дійсних числа ---
координати Б, пропускається рядок, через пробіл --- координати точки С.
Підрахувати площу трикутника. (Вказівка: напишіть функції підрахунку
довжини відрізка та функції обчислення площі трикутника за довжинами
сторін)
\item
  Напишіть власні функції, що обчислюють наступні вирази та відповідні
  власні функції, що будуть рахувати похідні даних функцій(Приклад,
  функція \(f(x) = identity(x) = x\), її похідна
  \(g(x) = \textrm{identity\_derivative}(x) = 1\)) :


  \begin{enumerate}[label=\xslalph*)]
  \item   \(f(x) = th(x) = \frac{(e^{x} - e^{-x})}{(e^{x} + e^{-x})}\);
\item \(f(x) = bent(x) = \frac{\sqrt{x^{2} + 1} - 1}{2} + x\);
\item \(f(x) = softSign(x) = \frac{x}{1 + |x|}\);
\item \(f(x) = arctg(x) = tg^{-1}(x)\);
\item\(f(x) = gauss(x) = e^{-x^{2}}\);
\item \(f(x) = softPlus(x) = \ln(1 + e^{x})\);
\item \(f(x) = sigmoid(x) = {(1 + e^{-x})}^{-1}\);
\item \(f(x) = invsqrt(x,\alpha) = \frac{x}{\sqrt{1 + \alpha x^{2}}}\);
\item\(f(x) = sigmweight(x) = x*{(1 + e^{-x})}^{-1}\).

\item

Перестановки

Вивести усі перестановки з N елементів у лексикографічному порядку.

Для 3: 123, 132, 213, 231, 312, 321 

\item
Вивести усі комбінації потужності K з множини розміру N .
Приклад:

3 2

Результат:

1 2
1 3
2 1
2 3
3 1
3 2 

 \end{enumerate}



\end{enumerate}


  \section{ 2. Оператори та типи даних Java. Умовні конструкції. Цикли}
\subsection{Прості типи}
\begin{enumerate}
\def\labelenumi{1.\arabic{enumi}.}
\item
Напишіть програму Java для відображення значення за замовчуванням усіх примітивних типів даних Java.
\item
Напишіть функцію перевірки, що два рядки рівними чи ні.
\item
Напишіть програму, яка використовує print та printf форму виведення.
\item
Створіть клас, що містить float, і використовуйте його для демонстрації псевдонімів
\item
Створіть клас, що містить float, і використовуйте його для демонстрації псевдонімів під час викликів методів.

\item
Напишіть програму, яка імітує підкидання монет.
\item
Покажіть, що шістнадцяткові та вісімкові визначення працюють з довгими значеннями. Використовуйте Long.toBinaryString() для відображення результатів.

\item
 Напишіть метод, який приймає два аргументи String і використовує всі булеві порівняння для порівняння двох рядків і друкує результати. Для == та !=. Також виконайте тест equals(). У main() викликайте свій метод з різними об'єктами String.


\end{enumerate}
\subsection{Бітові операції}
\begin{enumerate}
\def\labelenumi{2.\arabic{enumi}.}
\item
Напишіть програму, яка ініціалізує два цілих числа, одне задане з першим бітом рівним нулю, а інше, з одиницею у першому біті (підказка: Найпростіше використовувати шістнадцяткову систему для цього або бінарну). Візьміть ці два значення та виконайте всі
можливі операції з ними, використовуючи побітові оператори, і виведіть результати за допомогою Integer.toBinaryString().
\item
Використовуючи знаковий оператор зсуву вправо, змістіть його вправо до кінця через усі його двійкові позиції, кожен раз відображаючи результат за допомогою Integer.toBinaryString().
 Почніть з будь якого числа. Зсуньте його вліво, а потім використовуйте беззнаковий оператор зсуву вправо для переміщення вправо через усі його двійкові позиції, кожен раз відображаючи результат за допомогою Integer.toBinaryString().
\item
 Напишіть метод, який відображає значення char у двійковому вигляді. Продемонструйте це, використовуючи декілька різних символів.

\item
 Ввести цілі числа n і m вивести ціле число, у якого m-й біт відрізняється від m-го біта числа n, а всі інші біти збігаються з бітами числа n на тих же позиціях. Наприклад, якщо введені 5 і 1, відповіддю буде 7.

\item Ввести натуральне однобайтове число n (однобайтове число) і вивести число, отримане в результаті циклічного зсуву числа n на один розряд вліво, тобто старший біт посунутий в позицію молодшого, а всі інші біти зсуваються на один розряд вліво. Наприклад, якщо введено 130, відповіддю буде 5.

\item
 Напишіть клас, який розміщує двійкове представлення цілого натурального числа n у рядковій змінній s, де замість 0 буде символ крапка. 

\item Визначити, скільки разів зустрічається дана послідовність 0 та 1 (задана однобайтовим числом) в двійковому поданні цілого додатнього числа (так, в двійковому поданні 11010111 послідовність 11 зустрічається 3 рази, а послідовність 10 - 2 рази).
\item Викреслити $k$-й біт з двійкового представлення цілого числа (молодші $k$ бітів залишаються на місці, старші зсуваються на один розряд вправо). Наприклад, якщо введені 11 і 2, відповіддю буде 7.
\item Введіть довге натуральне число $m$ та натуральне $k$ типу байт та знайдіть $k$-тий розряд числа $m$.
\item Ввести натуральне число M. Встановіть її біт з номером j рівним нулеві та виведіть отримане число виведіть отримане число в десятковому та шістнадцятковому вигляді.
\item Визначить номер першого значущого зліва та зправа біта цілого числа M.
\item Поміняйте місцями перші 8 біт та останні 8 біт натурального числа, виведіть отримане число в десятковому та шістнадцятковому вигляді..
Ввести натуральне число M. Поміняйте місцями біти її двійкового запису з номерами i та j (що теж вводяться) та виведіть отримане число в десятковому та шістнадцятковому вигляді.

\item Знайдіть кількість значущих (не рівних 0) бітів цілого числа.

\item Ввести натуральні числа M та N та визначить скільки в них спільних одиничок бітового представлення. Визначить скільки в цих числах взагалі співпадає бітів.

\item Інвертуйте бітове представлення даного числа та виведіть двійкове представлення та десяткове для цієї інверсії.

\item
 Дано три натуральних числа: 32-бітне число, 8-бітний номер біта та булеве значення біта.
Необхідно поміняти в першому числі біт з заданим номером на передане значення та
вивести отримане після змін біта число в десятковій, шістнадцятковій та двійковй формі.

Приклад 1:
12  1 1 \\
Результат 1:
13  0xD 1101\\
Приклад 2:
13  1 0\\
Результат 2:
12  0xС 1100


\item Control Bit

На вхід подається ціле число $N $ ($1\le N \le 2^{28}-1 $).
Напишіть функцію, що повертає це число в звичайному форматі (big-endian) але
кожен байт числа має молодший біт, що визначається іншими 7 бітами 
як сума по модулю 2  значень кожного з цих 7 бітів (так званий "контрольний біт").
Напишить також функцію яка по довільному 32-бітному числу повертає відповідне 28-бітне число
або повідомлення, що контрольний біт на якомусь байті некоректний.

\item

Little-Endian, Big-Endian

У сучасному світі будь-які дані в пам'яті комп'ютера і при їх передачі по каналах зв'язку представлені у вигляді послідовності байтів. У різних системах і протоколах байти прийнято упорядковувати по-різному: від старшого до молодшого (big-endian) або від молодшого до старшого (little-endian).

Наприклад:

$2017_{10} = 11111100001_{2}$

Big-endian: $ \quad 2017_{10} = 00000111 \quad 11100001_{2}$
Little-endian:$ \quad 2017_{10} = 11100001\quad 00000111_{2} $

В блокчейн різницю між кодуванням чисел можна побачити: Bitcoin використовує little-endian, Ethereum використовує big-endian.

Напишіть програму, яка перетворює число до формату з little-endian з урахуванням того для гіпотетичної обчислювальної системи, де використовуються байти складаються тільки з 7 біт.

Формат вхідних даних:
На вхід подається ціле число $N $ ($1\le N \le 2^{28}-1 $).

Формат вихідних даних:
Необхідно вивести послідовність з 28 символів 0 або 1 (біти) - представлення вхідного числа в форматі little-endian.


\end{enumerate}

\subsection{Умови}

\begin{enumerate}
\def\labelenumi{3.\arabic{enumi}.}

\item
Напишіть програму, яка генерує 5 випадкових значень int. 
Введить дійсне число $a$ та за допомогою оператору if-else введіть всі числа, які більше та всі числа які менше $a$.

\item
Напишіть програму AllEqual.java, яка приймає чотири цілочисельні аргументи командного рядка і 
друкує пари індексів всіх пар рівних елементів. Наприклад, 1 1 3 3 друкує 1 2, 3 4, а 1 1 2 1 друкує 1 2, 1 4, 2 4.

\item
Напишить функцію, яка приймає у якості першого аргументу число яке представляє довжину, 
а якості другого --- величину виміру як рядок вигляду "мм", "см", "км" і т.п. 
Функція повинна правильно виводити довжину в метрах.

\item Напишить програму, що вводить два довгих цілих числа та додає та віднімає їх уникаючи проблему переповнення типу.
\item Напишить програму, що вводить два цілих числа та виконує арифметичні дії над ними (+,-,*,/,\%) таким чином, як це повинно виконуватися для беззнакового (натурального) типу.

\item Напишіть програму, яка бере три цілі аргументи командного рядка a, b і c і надрукуйте кількість різних значень (1, 2 або 3) серед a, b і c.

\item Напишіть програму, яка бере п'ять цілочисельних аргументів командного рядка та друкує медіану (третю за величиною).
Спробуйте обчислити медіану 5 елементів так, щоб після виконання вона ніколи не робила більше 6 загальних порівнянь.

\item Напишить клас DayOfWeek.java так, щоб він виводив назву дня тижня яке задано днем з початку року.
Використовуйте оператор switch.
\item Число українською.
Напишіть програму для зчитування в цілому числі командного рядка між -999,999,999 та 999,999,999 та виведить літерний еквівалент.
 Ось вичерпний перелік слів, які має використовувати ваша програма:
 мінус, нуль, один, два, три, чотири, п’ять, шість, сім, вісім, дев’ять, десять, одинадцять, дванадцять, тринадцять, чотирнадцять, п’ятнадцять, шістнадцять, сімнадцять , вісімнадцять, дев’ятнадцять, двадцять, тридцять, сорок, п’ятдесят, шістдесят, сімдесят, вісімдесят, дев’яносто, сто, тисяча, мільйон.

\item Оцінка гімнастки визначається колегією з 6 суддів, кожен з яких визначає оцінку від 0.0 до 10.0. 
(З точністю до 0.1). Остаточний бал визначається, відкидаючи високі та низькі бали та усереднюючи решту 4.
Напишіть програму GymnasticsScorer.java, яка приймає 6 дійсних вхідних даних командного рядка, що представляють 6 балів,
перевірте коректність даних та роздрукуйте отриману оцінку. 

\item Скласти програму, яка по колу та пpямій  встановлює, який випадок має місце:
      а) дві точки пеpетину;
      б) одна точка дотику;
      в) жодної спільної точки.

\item Задано два квадрати, сторони яких паралельні координатним осям. З'ясувати, чи перетинаються вони. Якщо так, то знайти координати лівого нижнього та правого верхнього кутів прямокутника, що є їхнім перетином.
\item Дано два прямокутники, сторони яких паралельні координатним осям. Відомо координати лівого нижнього та правого верхнього кутів кожного з прямокутників. Знайти координати лівого нижнього та правого верхнього кутів мінімального прямокутника, що містить задані прямокутники.

\item Ввести число от 1  до 12 та число від 1 до 31 та число від 2000 до 2100. Перевірити, що це коректна дата та коректний ввод. Вивести цю дату з назвою місяця українською.

\item Написати програму для обчислення коренів квадратних рівнянь у дійсних та комплексних числах (тип результату повинен визначати користувач). 

\item Створити клас, що розв'язує систему 2-х лінійних рівнянь в дійсних числах.

\item Створити клас, що розв'язує біквадратне рівняння в комплексних числах.

\end{enumerate}

\subsection{Цикли}

\begin{enumerate}
\def\labelenumi{4.\arabic{enumi}.}

\item Напишіть програму FivePerLine.java, яка за допомогою одного циклу
 for та одного оператора if друкує цілі числа від 1000 до 2000 з п’ятьма цілими числами на рядок. Підказка: використовуйте оператор \%.

\item
Напишіть програму, яка друкує значення від 1 до заданого числа n. При цьому ці числа виводяться через коми не більше ніж по п'ять
чисел у кожному рядку, при цьому рядки повинні бути йти ровними колоннами (тобто однакова кількість символів на число).

\item
 Напишіть програму ThreeLargest.java, яка зчитує цілі числа зі стандартного вводу до першого нуля та роздруковує три найбільші з них.

\item Таксі Рамануджана. С. Рамануджан був індійським математиком, який прославився своєю інтуїцією щодо чисел.
 Коли одного разу англійський математик Г. Х. Харді приїхав до нього в госпіталь, Харді зауважив, що номер його таксі - 1729,
 досить нудне число. На що Рамануджан відповів: "Ні, Харді! Ні, Харді! Це дуже цікаве число.
Це найменше число, яке можна виразити як суму двох кубів двома різними способами".
Перевірте це твердження, написавши програму Ramanujan.java, яка приймає цілий аргумент командного рядка $n$ і друкує всі цілі числа,
 менші або рівні $n$, які можна виразити як суму двох кубів двома різними способами - знайдіть різні додатні цілі числа a, b, c і d такі,
 що $a^{3} + b^{3} = c^{3} + d^{3}$. 
Тепер номер 87539319 здається досить нудним номером. Визначте, чому це не так.

\item   Гіпотеза Ейлера.
У 1769 р. Леонард Ейлер сформулював узагальнену версію Останньої теореми Ферма, висловивши припущення, що для отримання суми,
 яка сама по собі є $n$-ою степенню, потрібні принаймні $n$ $n$-тих степенів, для $n>2$.
Напишіть програму Euler.java для спростування гіпотези Ейлера (яка проіснувала до 1967 р.),
для пошуку чотирьох натуральних чисел, чия 5 -я ступінь дорівнює 5 -й степені іншого натурального числа.
Тобто знайдіть $a$, $b$, $c$, $d$ і $e$ такі, що $a^{5} + b^{5} + c^{5} + d^{5} = e^{5}$.
Використовуйте довгий тип даних.


\item Десяткове розкладання раціональних чисел.
Враховуючи два цілих числа $p$ і $q$, десятковий розклад $p/q$ має нескінченно повторюваний цикл.
Наприклад, $1/33 = 0.03030303\ldots$. Ми використовуємо позначення $0.(03)$ щоб вказати, що $03$ повторюється нескінченно довго.
Інший приклад, $8639/70000 = 0.1234(142857)$.
Напишіть програму DecimalExpansion.java, яка зчитує два цілих числа командного рядка $p$ і $q$ та друкує десятковий розклад $p/q$,
 використовуючи зазначені вище позначення. 

\item У п’ятницю 13-го. Яка максимальна кількість днів поспіль, у які не буває п’ятниці 13-го?
Підказка: Григоріанський календар повторюється кожні 400 років (146097 днів), тому вам потрібно турбуватися лише про 400-річний інтервал.
Результат: 426 (наприклад, з 13.08.1999 по 13.10.2000).
\item
 1 січня. Чи більше шансів, що 1 січня випаде на суботу чи неділю?
Напишіть програму, щоб визначити, скільки разів кожен з них відбуватиметься з інтервалом 400 років.
Результат: неділя (58 разів) частіше, ніж субота (56 разів).
\item
 Напишіть програму, яка приймає цілий аргумент командного рядка n і друкує цілі числа від 1 до 999 з n чисел на рядок. Зробіть цілі числа вирівняними, надрукувавши правильну кількість пробілів перед числом (наприклад, три для 1-9, два для 10-99 і один для 100-999).

\item
 Знайдіть всі прості числа з заданого двома цілими числами інтервалу. 

\item Проблема опадів.
Напишіть програму Rainfall.java, яка читатиме невід’ємні цілі числа (що представляють кількість опадів)
по одному за раз, доки не буде введено 999999, а потім роздрукує середнє значення (не враховуючи 999999).

\item Видаліть дублікати.
Напишіть програму Duplicates.java, яка читає послідовність цілих чисел і роздруковує цілі числа,
за винятком того, що вона видаляє повторювані значення, якщо вони з’являються послідовно.
Наприклад, якщо введено 1 2 2 1 5 1 1 7 7 7 7 1 1 1, то виведеться 1 2 1 5 1 7 1.

\item Роздрукуйте випадкове слово. Прочитайте список із $N$ слів з консолі, де $N$ невідомо заздалегідь,
і роздрукуйте одне з $N$ слів випадково. Не зберігайте список слів.
Замість цього скористайтеся методом Кнута: під час читання $i$-го слова виберіть його з імовірністю $1/i$ як вибране слово,
замінивши попереднього чемпіона. Роздрукуйте слово, яке збережеться після прочитання всіх даних.

\item 
Знайдіть всі цілі числа Армстронга в даному інтервалі (числа Армстронга - ті що дорівнюють сумі свої цифр у ступені, що співпадає з розрядністю даного числа).

\item
Число "вампіра" має парну кількість цифр і формується шляхом множення пари чисел, що містять половину цифр результату. Цифри беруться з вихідного номера в довільній послідовності. Пари кінцевих нулів не допускаються. 
Приклади: 1260 = 21 * 60, 1827 = 21 * 87, 2187 = 27 * 81.

Напишіть програму, яка знайде всі 4-значні числа вампірів. 


\item Ввести натуральне число (перевірити його натуральність) та підрахувати його факторіал (рекурсивно та циклом). Для якого найбільшого числа програма буде працювати коректно при використанні примітивних типів? Перепишить програму за допомогою класів Java, що буде працювати для будь-якого цілого числа.

\item Ввести послідовність наступним чином: користувачу виводиться напис “a[**]= ”, де замість ** стоїть номер числа, що вводиться. Тобто там виводится написи “a[0]= ”, і після знаку рівності користувач вводить число,  “a[1]= ”,  і після знаку рівності користувач вводить число і так далі доки користувач не введе число 0. Після цього потрібно вивести суму введених чисел. 

\item Визначити із скількох від`ємних чисел починається ненульова послідовність цілих чисел, за якою іде 0.

\item Обчислити значення експоненти за рядом Тейлора для дійсного числа x (-20<x<20) та порівняйте зі значенням обчисленим з допомогою бібліотечної функції. 

\item
  Скласти програми наближеного обчислення суми всіх доданків, абсолютна
  величина яких не менше $\varepsilon > 0 $:
\begin{enumerate}[label=\xslalph*)]
\item \(y =  \sin x = x - \frac{x^{3}}{3!} + \frac{x^{5}}{5!} - \ldots\);
\item \(y = \cos x = 1 - \frac{x^{2}}{2!} + \frac{x^{4}}{4!} - \ldots\);
\item
\(y =  \sinh (x) = x + \frac{x^{3}}{3!} + \frac{x^{5}}{5!} + \ldots\);
\item 
\(y = \cosh (x) = 1 + \frac{x^{2}}{2!} + \frac{x^{4}}{4!} + \ldots\);
\item \(y = e^{x} = 1 + \frac{x}{1!} + \frac{x^{2}}{2!} + \ldots\);
\item
\(y = \ln(1 + x) = x - \frac{x^{2}}{2!} + \frac{x^{3}}{3!} - \ldots,(\left| x \right| < 1)\);
\item
\(y = \frac{1}{1 + x} = 1 - x + x^{2} - x^{3} + \ldots,(\left| x \right| < 1)\);
\item
\(y = \ln\frac{1 + x}{1 - x} = 2*\frac{x}{1} + \frac{x^{3}}{3} + \frac{x^{5}}{5} + \ldots, (\left| x \right| < 1)\);
\item
\(y = \frac{1}{(1 + x)^{2}} = 1 - 2*x + 3*x^{2} - \ldots,(\left| x \right| < 1)\);
\item
\(y = \frac{1}{(1 + x)^{3}} = 1 - \frac{2*3}{2}x + \frac{3*4}{2}x^{2} - \frac{4*5}{2}x^{3} + \ldots,(\left| x \right| < 1)\);
\item
\(y = \frac{1}{1 + x^{2}} = 1 - x^{2} + x^{4} - x^{6} + \ldots,(\left| x \right| < 1)\);
\item
\(y =   \sqrt{1 + x} = 1 + \frac{1}{2}x - \frac{1}{2*4}x^{2} + \frac{1*3}{2*4*6}x^{3} - \ldots,(\left| x \right| < 1)\);
\item
\(y = \frac{1}{ \sqrt{1 + x}} = 1 - \frac{1}{2}x + \frac{1*3}{2*4}x^{2} - \frac{1*3*5}{2*4*6}x^{3} - \ldots,(\left| x \right| < 1)\);
\item
\(y = \Phi (x) = \int\limits_{0}^{x} e^{-x^{2}} dx = x - \frac{x^{3}}{3 \cdot 1!} + \frac{ x^{5}}{5 \cdot 3!} + \ldots\);
\item
\(y = \arcsin (x) = x + \frac{1}{2}\frac{x^{3}}{3!} + \frac{1*3}{2*4}\frac{x^{5}}{5!} + \ldots,(\left| x \right| < 1)\).

\end{enumerate}

\emph{\emph{Вказівка}}. Суму $y$ обчислювати за допомогою
рекурентного співвідношення
\(S_{0} = 0,\ S_{k} = S_{k - 1} + a_{k},\ k = 1,2,\ldots,\) де
\(a_{k} - k\)-тий доданок, для обчислення якого також складається
рекурентне співвідношення. В якості умови повторення циклу розглядається
умова \(\left| a_{k} \right| \geq \varepsilon.\)

\item Дана непорожня послідовність різних дійсних чисел, серед яких є хоча б одне від`ємне число, за якою йде 0. Визначити величину найбільшого серед від`ємних членів цієї послідовності.
\item Банк пропонує річну ставку по депозиту A та 15\% по вкладу додаються до основної суми депозиту кожен рік. Ви кладете в цей банк D гривень. Скільки років потрібно чекати, щоб сума вкладу зросла  до очікуваної суми P? 


\item Напишіть програму RollDie.java, яка генерує результат кидання чесного шестигранного кубика (ціле число від 1 до 6).

\item  Напишіть програму RollLoadedDie.java, яка друкує результат кидання кубика так,
що ймовірність отримання 1, 2, 3, 4 або 5 дорівнює 1/8, а ймовірність отримання 6 - 3/8.

 \item Аліса кидає чесну монету, поки не побачить два послідовних орли.
Боб кидає ще одну чесну монету, поки не побачить орла, за яким йде решка.
Написати програму для оцінки ймовірності того, що Аліса зробить менше кидків, ніж Боб.
( Відповідь: 39/121).
\item
 Припустимо, що a, b і c - випадкові числа, рівномірно розподілені між 0 і 1.
 Яка ймовірність того, що a, b і c утворюють довжину сторони деякого трикутника?
Підказка: вони складуть трикутник тоді і тільки тоді, коли сума кожних двох значень більша за третю. Використайте метод Монте-Карло.

\item Повторіть попереднє питання, але обчисліть ймовірність того, що отриманий трикутник тупий, враховуючи, що три числа для трикутника.
Підказка: три довжини утворюють тупий трикутник тоді і тільки тоді, коли сума кожних двох значень більша за третю та сума квадратів кожних двох довжин сторін більша або дорівнює квадрату третього.

\item Ігрове моделювання.
У ігровому шоу «Давайте укладемо угоду» учаснику показують три двері.
За одними дверима цінний приз, за двома іншими - нічого.
Після того, як учасник обирає двері, ведучий відкриває одну з інших двох дверей (звичайно, ніколи не розкриваючи приз).
Після цього учаснику надається можливість перейти до інших невідчинених дверей.
Чи повинен це зробити учасник? Інтуїтивно може здатися, що двері першого вибору конкурсанта та інші невідкриті двері однаково ймовірно міститимуть приз, тому не буде стимулу змінювати думку.
Напишіть програму MonteHall.java, щоб перевірити цю інтуїцію шляхом моделювання.
 Ваша програма повинна взяти цілий аргумент командного рядка $n$, пограти в гру $n$ разів,
використовуючи кожну з двох стратегій (перемикання чи не перемикання) і надрукувати шанс на успіх для кожної стратегії.
Або ви можете пограти в гру тут.

\end{enumerate}


\end{document}


